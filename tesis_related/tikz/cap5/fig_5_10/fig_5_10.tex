\documentclass{article}
\usepackage{tikz}
\usepackage{verbatim}
\usepackage[active,tightpage]{preview}
\PreviewEnvironment{tikzpicture}
\setlength{\PreviewBorder}{15pt}%
\begin{comment}
\end{comment}
\usetikzlibrary{dsp,chains}

\DeclareMathAlphabet{\mathpzc}{OT1}{pzc}{m}{it}
\newcommand{\z}{\mathpzc{z}}

\begin{document}
\begin{tikzpicture}

				%\draw[help lines,step=0.5] (-5,-2) grid (5,2);
	% Place nodes using a matrix
	\matrix (m1) [row sep=3mm, column sep=5mm]
	{
		%first row
		%------------------------------------------------------------
		\node[coordinate]                  (m000) {};          &
		\node[coordinate]                  (m001) {};          &
		\node[dspsquare,minimum width=1.5cm](m002) {$h_0(n)$}; &
		\node[dspsquare]                   (m003) {$\downsamplertext{2}$}; &
		\node[coordinate]                  (m004) {};          &
		\node[coordinate]                  (m005) {};          \\
		%------------------------------------------------------------
		%second row
		%------------------------------------------------------------
		\node[dspnodeopen,dsp/label=above] (m100) {$x(n)$};    &
		\node[coordinate]                  (m101) {};          &
		\node[coordinate]                  (m102) {};          &
		\node[coordinate]                  (m103) {};          &
		\node[coordinate]                  (m104) {};          &
		\node[coordinate]                  (m105) {};          \\
%		%------------------------------------------------------------
		%third row
		%------------------------------------------------------------
		\node[coordinate]                  (m200) {};          &
		\node[coordinate]                  (m201) {};          &
		\node[dspsquare,minimum width=1.5cm](m202) {$h_1(n)$}; &
		\node[dspsquare]                   (m203) {$\downsamplertext{2}$}; &
		\node[coordinate]                  (m204) {};          &
		\node[coordinate]                  (m205) {};          \\ 
		%------------------------------------------------------------
	};

	% Draw connections
	%\foreach \i [evaluate= \i as \j using int(\i+2),
	%             evaluate= \i as \k using int(\i+3)] in {0,1,2,3}
	\foreach \i in {0,2}
	{
		\begin{scope}[start chain]
			\chainin (m\i01);
			\chainin (m\i02) [join=by dspconn];
			\chainin (m\i02);
			\chainin (m\i03) [join=by dspconn];
			\chainin (m\i03);
			\chainin (m\i04) [join=by dspconn];
		\end{scope}
	}
	
	\draw[dspflow] (m100) -- (m101);
	\draw[thick] (m101) |- (m001);
	\draw[thick] (m101) |- (m201);

	\node [right of=m004,right=-1.0cm,scale=1.0](o01){$y_0(m)$};
	\node [right of=m204,right=-1.0cm,scale=1.0](o02){$y_1(m)$};

	\
\end{tikzpicture}

\end{document}
