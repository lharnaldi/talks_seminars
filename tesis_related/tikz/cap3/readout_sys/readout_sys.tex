\documentclass[14pt,border=10pt]{standalone}
\usepackage{tikz}
\usepackage[american]{circuitikz}
\usepackage{textcomp}

\usetikzlibrary{shapes,arrows}

\begin{document}
% Definition of blocks:
\tikzset{%
  block/.style    = {draw, thick, rectangle, align=center, minimum height = 3em, minimum width = 3em},
  blockg/.style    = {draw, thick, rectangle, align=center, minimum height =
3cm, minimum width = 3cm},
  blockp/.style    = {draw, thick, dashed, rectangle, align=center, minimum height =
3cm, minimum width = 5cm},
  blockgg/.style    = {draw, thick, rectangle, align=center, minimum height =
3cm, minimum width = 2cm},
  mult/.style      = {draw, circle, node distance = 2cm}, % Adder
  input/.style    = {coordinate}, % Input
  output/.style   = {coordinate} % Output
}
% Defining string as labels of certain blocks.
\newcommand{\suma}{\Large$+$}
\newcommand{\mult}{\Huge x}
\newcommand{\inte}{$\displaystyle \int$}
\newcommand{\derv}{\huge$\frac{d}{dt}$}

\begin{tikzpicture}[auto, thick, node distance=2cm, >=triangle 45]

%\draw[help lines,step=0.5,dashed] (-1,-8) grid (28,4); 
\draw 
  % Drawing the blocks of first filter :
  %node at (0,0)[right=-3.3mm]{\Large \textopenbullet}
  node at (0,0)[input, name=sigin] {}
  %node at (0,-2)[right=-3.3mm]{\Large \textopenbullet}
  node at (0,-2)[input, name=refin] {}
  node at (2,0)[block,minimum width=2cm, right of=sigin] (da1) {D/A \\ 16 bits
	\\ 400\,MHz}
  node at (0,-2)[block,minimum width=2cm, right of=refin] (da2) {D/A \\ 16 bits
	\\ 400\,MHz}
  node [block, right of=da1] (lpf1i) {FPB}
  node at (5.8,0.3)[]{$I$}
  node at (4.8,-1)[]{Mezclador $IQ$}
  node [block, right of=da2] (lpf2i) {FPB}
  node at (5.8,-2.3)[]{$Q$}
  node at (6.5,-1) [mixer] (mult1) {}
  node at (6.8,-1.4)[input, name=synthin1] {}
  node at (13.5,-1)[mixer] (mult2) {}
  node at (15.2,-1)[]{Mezclador $IQ$}
  node at (13.2,-1.4)[input, name=synthin2] {}
  %node [block, right of=ad2] (zcd) {ZCD}
  %node [block, right of=zcd] (adpll) {ADPLL}
  %node [block, below of=adpll] (intref) {INT. REF.}
  node at (10,-1)[blockp] (crio) {}
  node at (10,0.2)[]{Crióstato}
	node at (9,-1)[block,minimum height=1.5cm,minimum width=2cm] (mkid)
	{Detector\\criogénico}
  node at (11.5,-1)[buffer] (gain) {}
	node at (10,-4)[block,minimum width=5cm] (synth) {Sintetizador\\Referencia
	(GHz)}
	node at (16,0)[block] (lpf1o) {FPB}
	node at (16,-2)[block] (lpf2o) {FPB}
  node at (18,0)[block] (ad1) {A/D \\ 14 bits \\ 400\,MHz}
  node at (18,-2)[block] (ad2) {A/D \\ 14 bits \\ 400\,MHz}
  node at (22,-1)[blockg] (fpga) {FPGA\\ BF-DFTG-SM \\ K canales}
  node at (22,-5)[blockg,minimum width=5cm] (pc) {PC\\Procesamiento \\ Memoria
	\\ Conexión
	USB/Ethernet}
  node at (20,0)[output, name=sigout] {}
  node at (20,-2)[output, name=refout] {};

   % Joining blocks. 
    % Commands \draw with options like [->] must be written individually
	\draw[-](da1) -- (lpf1i.west);
  \draw[->](lpf1i) -| (mult1.north);
	\draw[-](da2) -- (lpf2i.west);
  \draw[->](lpf2i) -| (mult1.south);
  \draw[->](mult1.east) -- node {} (mkid);
  \draw[->](mkid.east) -- (gain.west);
  \draw[->](gain.east) -- (mult2.west);
  \draw[->](mult2.north) |- (lpf1o.west);
  \draw[->](mult2.south) |- (lpf2o.west);
	\draw[-](lpf1o.east) -- (ad1);
	\draw[-](lpf2o.east) -- (ad2);
  \draw[->](ad1) -- node {} (20.5,0);
  \draw[->](ad2) -- node {} (20.5,-2);
  \draw[<->](fpga.south) -- (pc.north);
	\draw[->](synth.west) -| (synthin1);
	\draw[->](synth.east) -| (synthin2);
%  %\draw[->](refin) -- node {CH2} (ad2);
%  \draw[->](ad2) -- node {}(zcd);
%  \draw[->](zcd) -- node {}(adpll);
%  \draw[->](intref) -- node {} (adpll);
%  \draw[->](soc) |- node {} (intref);
%  %\draw[->](mult2.east) -- node {} (lpf2);
%  \draw[->](lpf2) -- node {} (12.5,-2);
%  \draw[*-] (4,0.1) -- node {} (4,-2);
%  \draw[->] (4,-2) -- (7.5,-2);
%  %\draw[->] (adpll.east) -| node {} (mult2.south);
%  \draw (adpll.north) -- node {} (6,-1);
%  %\draw[->] (6,-1) -| node {} (mult1.south);
%  \draw[->] (15.5,0) -- node {} (17,0);
%  \draw[->] (15.5,-2) -- node {} (17,-2);
%  \draw[->] (18.2,0) -- node {Out1} (20,0);
%  \draw[->] (18.2,-2) -- node {Out2} (20,-2);
  
\end{tikzpicture}
\end{document}
