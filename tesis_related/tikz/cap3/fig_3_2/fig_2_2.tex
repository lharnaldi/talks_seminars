\documentclass[14pt,border=10pt]{standalone}
\usepackage{tikz}
%\usepackage[american]{circuitikz}
\usepackage[]{circuitikz}
\usepackage{textcomp}

\usetikzlibrary{shapes,arrows}

\begin{document}
% Definition of blocks:
\tikzset{%
  block/.style    = {draw, thick, rectangle, align=center, minimum height = 3em, minimum width = 3em},
  blockg/.style    = {draw, thick, rectangle, align=center, minimum height =
3cm, minimum width = 3cm},
  blockp/.style    = {draw, thick, dashed, rectangle, align=center, minimum height =
3cm, minimum width = 5cm},
  blockgg/.style    = {draw, thick, rectangle, align=center, minimum height =
3cm, minimum width = 2cm},
  mult/.style      = {draw, circle, node distance = 2cm}, % Adder
  input/.style    = {coordinate}, % Input
  output/.style   = {coordinate} % Output
}
% Defining string as labels of certain blocks.
\newcommand{\suma}{\Large$+$}
\newcommand{\mult}{\Huge x}
\newcommand{\inte}{$\displaystyle \int$}
\newcommand{\derv}{\huge$\frac{d}{dt}$}

\begin{tikzpicture}[auto, thick, node distance=2cm, >=triangle 45]

%\draw[help lines,step=0.5,dashed] (-1,-8) grid (28,4); 
\draw 
  % Drawing the blocks of first filter :
  node at (0,0)[input, name=Iindac] {}
  node at (0,-2)[input, name=Qindac] {}
  node at (2,-4)[input, name=deriv] {}
	%node [block, right of=Iindac] (lpf1i) {FPB}
	node at (3,0) [right of=Iindac] (lpf1i) {}
	node at (2,0)[block,label=10\,dB,right of=lpf1i] (att1i) {Aten.}  
  node at (5.8,0.3)[label={[xshift=0.0cm, yshift=-0.4cm]0\,dBm}]{}
  %node at (4.8,-1)[]{Mezclador $IQ$}
  %node [block, right of=Qindac] (lpf2i) {FPB}
	node at (5,-2) [right of=Qindac] (lpf2i) {}
	node at (0,-2)[block,label=10\,dB,right of=lpf2i] (att2i) {Aten.}
  %node [block,below of=lpf2i] (lpf3i) {FPB}
	%node at (2,0)[block,right of=lpf2i] (da1) {D/A \\ 16 bits}
  %node [block,below of=lpf3i] (lpf5i) {FPB}
  node at (5.8,-2.3)[label={[xshift=0.0cm, yshift=-0.4cm]0\,dBm}]{}
  node at (6.5,-1) [mixer] (mult1) {}
	node at (10,-1)[block,label=0-30\,dB,right of=mult1] (varatt1) {Aten. Var.}
  %node at (8.4,-2.1)[]{0-30\,dB}
	node at (13,-1)[align=left](RFout){Salida RF \\(a crióstato) \\ 2-8\,GHz, -40
	a -10\,dBm}
	node at (-2,0)[align=left](Iin){Entrada $I$ \\(desde D/A) \\ 0-250\,MHz,+10\,dBm}
	node at (-2,-2)[align=left](Qin){Entrada $Q$ \\(desde D/A) \\ 0-250\,MHz,+10\,dBm}
	node at (-2,-4)[align=left](OLin){OL (2-8\,GHz, +16\,dBm) \\$<$
	-20\,dBc/Hz @ 1\,Hz \\ $<$ -40\,dBc/Hz @ 10\,Hz}
  node at (2,-7)[buffer,label={[xshift=0.0cm, yshift=0.8cm]30\,dB}] (gain1) {}
	node at (-2,-7)[align=left](RFin){Entrada RF \\(desde crióstato) \\ 2-8\,GHz,-45 a -25\,dBm}
  %node [block,label=0-30\,dB,right of=gain1] (att3i) {FPB}
	node at (2,-7) [right of=gain1,label={[xshift=0.0cm, yshift=0.5cm]0-30\,dB}] (att3i) {}
  node at (2,-7)[buffer,label={[xshift=0.0cm, yshift=0.8cm]30\,dB}, right of=att3i] (gain2) {}
  node at (7.2,-7)[label={[xshift=0.0cm, yshift=-0.1cm]0\,dBm}]{}
  node at (4,-7)[mixer,xshift=0.5cm,right of=gain2] (mult2) {}
  node at (11,-5)[buffer,label={[xshift=0.0cm, yshift=0.8cm]20\,dB}] (gain3) {}
  node at (11,-9)[buffer,label={[xshift=0.0cm, yshift=0.8cm]20\,dB}] (gain4) {}
	node [block,label=0-20\,dB,right of=gain3] (att4) {Aten. var.}
  node [block,label=0-20\,dB,right of=gain4] (att5) {Aten. var.}
	%node [block,right of=att4] (lpf6o) {FPB}
	node at (5,-5) [right of=att4] (lpf6o) {}
	%node at (5,-2) [right of=Qindac] (lpf2i) {}
  node at (0,-2)[input, name=Qindac] {}
	node at (19,-5)[align=left](Iadcout){Salida $I$ (al A/D)\\ 0-250\,MHz,+10\,dBm}
  %node [block,right of=att5] (lpf7o) {FPB}
	node at (5,-9) [right of=att5] (lpf7o) {}
	node at (19,-9)[align=left](Qadcout){Salida $Q$ (al A/D)\\ 0-250\,MHz,+10\,dBm}
  node at (20,-1)[output, name=dacout] {}
  node at (20,-2)[output, name=Ioutdac] {}
  node at (20,-4)[output, name=Qoutdac] {};
	\path (Iindac) to [lowpass,name=lpf1ii] (att1i);
	\path (Qindac) to [lowpass,name=lpf2ii] (att2i);
	\path (gain1) to [lowpass,name=att3ii] (gain2);
	\path (att4) to [lowpass,name=lpf6oo] (Iadcout);
	\path (att5) to [lowpass,name=lpf7oo] (Qadcout);

   % Joining blocks. 
    % Commands \draw with options like [->] must be written individually
  \draw[->](Iin) -- (lpf1ii);
  \draw[->](Qin) -- (lpf2ii);
  \draw[-](mult1.east) -- (varatt1);
  \draw[->](varatt1.east) -- (RFout);
  \draw[->](OLin) -- (deriv);
	\draw[->](deriv) -| (5,-1) -- (mult1.west);
	\draw[->](deriv) -| (9.5,-7) -- (mult2.east);
	\draw[-](lpf1ii) -- (att1i);
  \draw[->](att1i) -| (mult1.north);
  \draw[-](lpf2ii) -- (att2i);
	\draw[->](att2i) -| (mult1.south);
	%\draw[->](lpf6o) -- (Iadcout);
	%\draw[->](lpf7o) -- (Qadcout);
	\draw[->](lpf7oo) -- (Qadcout);
  \draw[->](RFin) -- (gain1.west);
  \draw[-](gain1.east) -- (att3ii);
  \draw[-](att3ii.east) -- (gain2.west);
  \draw[->](gain2.east) -- (mult2.west);
  \draw[->](mult2.north) |- (gain3.west);
  \draw[->](mult2.south) |- (gain4.west);
	\draw[-](gain3.east) -- (att4);
	\draw[-](gain4.east) -- (att5);
	\draw[-](att4.east) -- (lpf6oo);
	\draw[->](lpf6oo.east) -- (Iadcout);
	\draw[-](att5.east) -- (lpf7oo);
  
\end{tikzpicture}
\end{document}
