\documentclass[14pt,border=10pt]{standalone}
\usepackage{tikz}
\usepackage[american]{circuitikz}
\usepackage{textcomp}

\usetikzlibrary{shapes,arrows,dsp,chains}

\begin{document}
% Definition of blocks:
\tikzset{%
  block/.style    = {draw, thick, rectangle, rounded corners, fill={rgb:black,1;white,10},align=center, minimum height = 3em, minimum width = 3em},
  blockg/.style    = {draw, thick, rectangle, align=center, minimum height = 3cm, minimum width = 3cm},
  blockgg/.style    = {draw, thick, rectangle, align=center, minimum height = 3cm, minimum width = 2cm},
  mult/.style      = {draw, circle, node distance = 2cm}, % Adder
  input/.style    = {coordinate}, % Input
  output/.style   = {coordinate} % Output
}
% Defining string as labels of certain blocks.
\newcommand{\suma}{\Large$+$}
\newcommand{\mult}{\Huge x}
\newcommand{\inte}{$\displaystyle \int$}
\newcommand{\derv}{\huge$\frac{d}{dt}$}

\begin{tikzpicture}[auto, thick, node distance=2cm, >=triangle 45]

%\draw[help lines,step=0.5cm] (-1,-8) grid (24,20); 
				\draw [block, align=center,name=cfilt] (3.5,8) rectangle (7.5,10);
				\draw [align=center,name=cfilt,dashed] (3.5,4.2) rectangle (7.5,7.3);
				\draw [align=center,name=cfilt,dashed] (3.5,0.7) rectangle (7.5,3.8);
				\draw [block, name=I1] (4,6) rectangle (7,7);
				\draw [block, name=I2] (4,4.5) rectangle (7,5.5);
				\draw [block, name=Q1] (4,2.5) rectangle (7,3.5);
				\draw [block, name=Q2] (4,1) rectangle (7,2);
				%nombres
				\draw node at (5.5,9) [scale=1]{$H(z)$ Complejo};
				\draw node at (5.5,6.5) [scale=1]{$I(z)$};
				\draw node at (5.5,5) [scale=1]{$I(z)$};
				\draw node at (5.5,3) [scale=1]{$Q(z)$};
				\draw node at (5.5,1.5) [scale=1]{$Q(z)$};
				%sumas
				\node at (9,6.5) [dspadder,scale=1.3] {};
				\node at (9,1.5) [dspadder,scale=1.3] {};

				%ahora las entradas
				\draw node at (2,9)[name=ine00]{\Large \textopenbullet};
				\draw node at (2,6.5)[name=ine01]{\Large \textopenbullet};
				\draw node at (2,1.5)[name=ine02]{\Large \textopenbullet};
				%conexiones de entrada
				\draw [->] (2.1,9) -- (3.5,9);
				\draw [->] (2.1,6.5) -- (4,6.5);
				\draw [->] (2.5,6.5) |- (4,5);
				\draw node at (2.5,6.5) [circle,scale=0.3,fill=black] {\textopenbullet}; 
				\draw [->] (2.5,1.5) |- (4,3);
				\draw node at (2.5,1.5) [circle,scale=0.3,fill=black] {\textopenbullet}; 
				\draw [->] (2.1,1.5) -- (4,1.5);
				%conexiones de salida
				\draw [->] (7.5,9) -- (10.5,9);
				\draw [->] (7,6.5) -- (8.75,6.5);
				\draw [->] (7,1.5) -- (8.75,1.5);
				%salidas
				\draw [-] (7,3) -- (8,3);
				\draw [->] (8,3) -- (9,6.25);
				\draw [-] (7,5) -- (8,5);
				\draw [->] (8,5) -- (9,1.75);
				\draw [->] (9.25,6.5) -- (11,6.5);
				\draw [->] (9.25,1.5) -- (11,1.5);
				%%labels
				\node [left,scale=1.7] at (2,9) {$x(n)$};
				\node [right,scale=1.7] at (11,9) {$y(n)$};
				\node [left,scale=1.7] at (2,6.5) {$x_i(n)$};
				\node [right,scale=1.7] at (11,6.5) {$y_i(n)$};
				\node [left,scale=1.7] at (2,1.5) {$x_q(n)$};
				\node [right,scale=1.7] at (11,1.5) {$y_q(n)$};
				\node [above,scale=1.3] at (8.3,6.5) {$+$};
				\node [below,scale=1.3] at (8.6,6.2) {$-$};
				\node [above,scale=1.3] at (8.3,1.5) {$+$};
				\node [below,scale=1.3] at (9.3,2.4) {$+$};
				%nodos
				\node at (1,10) [scale=1.7] {\textbf{a)}};
				\node at (1,7.5) [scale=1.7] {\textbf{b)}};
  
\end{tikzpicture}
\end{document}
