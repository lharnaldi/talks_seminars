\documentclass[14pt,border=10pt]{standalone}
\usepackage{tikz}
\usepackage[american]{circuitikz}
\usepackage{textcomp}

\usetikzlibrary{shapes,arrows,dsp,chains}

\begin{document}
% Definition of blocks:
\tikzset{%
  block/.style    = {draw, thick, rectangle, rounded corners, fill={rgb:black,1;white,10},align=center, minimum height = 3em, minimum width = 3em},
  blockg/.style    = {draw, thick, rectangle, align=center, minimum height = 3cm, minimum width = 3cm},
  blockgg/.style    = {draw, thick, rectangle, align=center, minimum height = 3cm, minimum width = 2cm},
  mult/.style      = {draw, circle, node distance = 2cm}, % Adder
  input/.style    = {coordinate}, % Input
  output/.style   = {coordinate} % Output
}
% Defining string as labels of certain blocks.
\newcommand{\suma}{\Large$+$}
\newcommand{\mult}{\Huge x}
\newcommand{\inte}{$\displaystyle \int$}
\newcommand{\derv}{\huge$\frac{d}{dt}$}

\begin{tikzpicture}[auto, thick, node distance=2cm, >=triangle 45]

%\draw[help lines,step=0.5cm] (-1,-8) grid (24,20); 
				\draw [align=center,name=cfilt,dashed] (1.1,-0.5) rectangle (14.3,6);
				\draw [align=center,name=cfilt] (12,0) rectangle (14,5.5);
				\draw [align=center,name=del1] (2,4.5) rectangle (3,5.5);
				\draw [align=center,name=del2] (2,3) rectangle (3,4);
				\draw [block, name=fir0] (4,3) rectangle (7,5.5);
				\draw [block, name=Q1] (8,4.5) rectangle (10,5.5);
				\draw [block, name=Q2] (8,3) rectangle (10,4);
				\draw [block, name=fir1] (4,0) rectangle (7,2.5);
				\draw [block, name=Q1] (8,1.5) rectangle (10,2.5);
				\draw [block, name=Q2] (8,0) rectangle (10,1);
				%nombres
				\draw node at (2.5,5) [scale=1]{$z^{-M}$};
				\draw node at (2.5,3.5) [scale=1]{$z^{-M}$};
				\draw node at (5.5,4.3) [scale=1.2]{FIR 0};
				\draw node at (5.5,1.3) [scale=1.2]{FIR 1};
				\draw node at (9,5) [scale=1]{FIFO};
				\draw node at (9,3.5) [scale=1]{FIFO};
				\draw node at (9,2) [scale=1]{FIFO};
				\draw node at (9,0.5) [scale=1]{FIFO};
				\draw node at (13,3) [scale=1,align=center]{Máquina \\ de \\ estados};
				%ahora las entradas
				%\draw node at (2,9)[name=ine00]{\Large \textopenbullet};
				\draw node at (0.5,5)[name=ine01]{\Large \textopenbullet};
				\draw node at (0.5,3.5)[name=ine02]{\Large \textopenbullet};
				%conexiones de entrada
				%\draw [->] (2.1,9) -- (3.5,9);
				\draw [->] (0.6,5) -- (2,5);
				\draw [->] (1.6,5) |- (4,2);
				\draw node at (1.6,5) [circle,scale=0.3,fill=black] {\textopenbullet}; 
				\draw [->] (3,5) -- (4,5);
				\draw [->] (7,5) -- (8,5);
				\draw [->] (0.6,3.5) -- (2,3.5);
				\draw [->] (1.4,3.5) |- (4,0.5);
				\draw node at (1.4,3.5) [circle,scale=0.3,fill=black] {\textopenbullet}; 
				\draw [->] (3,3.5) -- (4,3.5);
				\draw [->] (7,3.5) -- (8,3.5);
				\draw [->] (7,2) -- (8,2);
				\draw [->] (7,0.5) -- (8,0.5);
				\draw [->] (10,5) -- (12,5);
				\draw [->] (10,3.5) -- (12,3.5);
				\draw [->] (10,2) -- (12,2);
				\draw [->] (10,0.5) -- (12,0.5);
				%salidas
				\draw [->] (14,5) -- (15,5);
				\draw [->] (14,3.5) -- (15,3.5);
				%%labels
				\node [left,scale=1.7] at (0.5,5) {$I$};
				\node [right,scale=1.7] at (15,5) {$I$};
				\node [left,scale=1.7] at (0.5,3.5) {$Q$};
				\node [right,scale=1.7] at (15,3.5) {$Q$};
				\node [above,scale=1] at (11.5,5) {$I_1$};
				\node [above,scale=1] at (11.5,3.5) {$Q_1$};
				\node [above,scale=1] at (11.5,2) {$I_2$};
				\node [above,scale=1] at (11.5,0.5) {$Q_2$};
  
\end{tikzpicture}
\end{document}
