\documentclass{article}
\usepackage{tikz}
\usepackage{verbatim}
\usepackage[active,tightpage]{preview}
\PreviewEnvironment{tikzpicture}
\setlength{\PreviewBorder}{15pt}%
\begin{comment}
\end{comment}
\usetikzlibrary{dsp,chains}

\DeclareMathAlphabet{\mathpzc}{OT1}{pzc}{m}{it}
\newcommand{\z}{\mathpzc{z}}

\begin{document}
\begin{tikzpicture}

				%\draw[help lines,step=0.5] (-5,-2) grid (5,2);
	% Place nodes using a matrix
	\matrix (m1) [row sep=6mm, column sep=5mm]
	{
		%------------------------------------------------------------
		\node[dspnodeopen,dsp/label=above] (m00) {$x_\rho(r)$};    &
		%\node[coordinate]                  (m01) {};          &
		%\node[dspmixer]                    (m02) {};          &
		\node[coordinate]                  (m03) {};          &
				\node[dspsquare]                   (m04) {\upsamplertext{I}};    &
		%\node[coordinate]                  (m05) {};          &
				\node[above]                  (m05) {};          &
		\node[dspsquare,minimum width=1.8cm]                   (m06) {$\bar{p}_\rho(m)$}; &
		\node[coordinate]                  (m07) {};          &
%		\node[coordinate]                  (m08) {};          &
		\node[dspnodeopen,dsp/label=above] (m09) {$y_\rho(m)$};  \\
		%------------------------------------------------------------
%		\node[coordinate]                  (m10) {};          &
%		%\node[coordinate]                  (m11) {};          &
%		%\node[dspnodeopen,dsp/label=below] (m12) {$e^{-j\frac{2\pi}{K}kn}$};    &
%		\node[coordinate]                  (m13) {};          &
%		\node[coordinate]                  (m14) {};          &
%		\node[coordinate]                  (m15) {};          &
%		\node[coordinate]                  (m15b) {};          &
%		\node[coordinate]                  (m16) {};          &
%		\node[coordinate]                  (m17) {};          &
%		\node[coordinate]                  (m18) {};          &
%		\node[coordinate]                  (m19) {};          \\
%		%------------------------------------------------------------
%		\\
%		%------------------------------------------------------------
%		\node[coordinate]                  (m20) {};          &
%		\node[coordinate]                  (m21) {};          &
%		\node[coordinate]                  (m22) {};          &
%		\node[coordinate]                  (m23) {};          &
%		\node[dspadder]                    (m24) {};          &
%		\node[coordinate]                  (m25) {};          &
%		\node[dspadder]                    (m26) {};          &
%		\node[coordinate]                  (m27) {};          &
%		\node[dspadder]                    (m28) {};          &
%		\node[coordinate]                  (m29) {};          &
%		\node[dspnodeopen,dsp/label=above] (m2X) {$y[n]$};    \\
		%--------------------------------------------------------------------
	};

	% Draw connections
	
	\begin{scope}[start chain]
		\chainin (m00);
		%\chainin (m02) [join=by dspflow];
		%\chainin (m12);
		%\chainin (m02) [join=by dspconn];
		%\chainin (m02);
		\chainin (m04) [join=by dspconn];
		\chainin (m04);
		\chainin (m06) [join=by dspconn];
	\end{scope}

	%\foreach \i [evaluate = \i as \j using int(\i+1),
	%             evaluate = \i as \k using int(\i+2),] in {2,4,6}
	%{
	%	\begin{scope}[start chain]
	%		\chainin (m0\i);
	%		\chainin (m0\j) [join=by dspconn];
	%		\chainin (m0\k) [join=by dspline];
	%		\chainin (m1\k) [join=by dspconn];
	%		\chainin (m2\k) [join=by dspconn];
	%	\end{scope}
	%	\draw[dspconn] (m2\i) -- (m2\k);
	%}

	\draw[dspflow] (m06) -- (m09);

\end{tikzpicture}

\end{document}
