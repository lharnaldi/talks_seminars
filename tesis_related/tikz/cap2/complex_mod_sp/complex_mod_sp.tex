\documentclass{article}
\usepackage{tikz}
\usepackage{verbatim}
\usepackage[active,tightpage]{preview}
\PreviewEnvironment{tikzpicture}
\setlength{\PreviewBorder}{15pt}%
\begin{comment}
\end{comment}
\usetikzlibrary{dsp,chains}

\DeclareMathAlphabet{\mathpzc}{OT1}{pzc}{m}{it}
\newcommand{\z}{\mathpzc{z}}

\begin{document}
\begin{tikzpicture}

				%\draw[step=0.5,very thin,black!20] (-5,-0.5) grid (5,1);
				%\draw[help lines] (-5,-2) grid (5,2);
	% Place nodes using a matrix
	\matrix (m1) [row sep=6mm, column sep=5mm]
	{
		%------------------------------------------------------------
		\node[dspnodeopen,dsp/label=above] (m00) {$x(n)$};    &
		\node[coordinate]                  (m01) {};          &
		\node[dspsquare,minimum height=2.3em, minimum width=3.1cm] (m02) {$h_k(n) = h(n)W_K^{-kn} $};    &
		\node[dspsquare,minimum height=2.3em, minimum width=1.3cm] (m03) {\downsamplertext{M=K}}; &
		\node[dspnodeopen,dsp/label=above] (m04) {$X_k(m)$};  \\
		%------------------------------------------------------------
		\node[coordinate]                  (m10) {};          &
		\node[coordinate]                  (m11) {};          &
		\node[coordinate]                  (m12) {};          &
		\node[coordinate]                  (m13) {};          &
		\node[coordinate]                  (m14) {};          \\
%		%------------------------------------------------------------
		\\
%		%------------------------------------------------------------
				\node[dspnodeopen,dsp/label=above] (m20) {$\hat{X}_k(m)$};    &
		\node[coordinate]                  (m21) {};          &
		\node[dspsquare,minimum height=2.3em, minimum width=1.3cm] (m22) {\upsamplertext{M=K}}; &
		\node[dspsquare,minimum height=2.3em, minimum width=3.1cm] (m23) {$f_k(n) = f(n)W_K^{-kn} $};    &
				\node[dspnodeopen,dsp/label=above] (m24) {$\hat{x}_k(n)$};  \\
		%------------------------------------------------------------
		\node[coordinate]                  (m30) {};          &
		\node[coordinate]                  (m31) {};          &
		\node[coordinate]                  (m32) {};          &
		\node[coordinate]                  (m33) {};          &
		\node[coordinate]                  (m34) {};          \\
		%--------------------------------------------------------------------
	};

	% Draw connections
	
	\begin{scope}[start chain]
		\chainin (m00);
		\chainin (m02) [join=by dspconn];
		\chainin (m02);
		\chainin (m03) [join=by dspconn];
		\chainin (m03);
		\chainin (m04) [join=by dspconn];
	\end{scope}
	\begin{scope}[start chain]
		\chainin (m20);
		\chainin (m22) [join=by dspconn];
		\chainin (m22);
		\chainin (m23) [join=by dspconn];
		\chainin (m23);
		\chainin (m24) [join=by dspconn];
	\end{scope}

	%\foreach \i [evaluate = \i as \j using int(\i+1),
	%             evaluate = \i as \k using int(\i+2),] in {2,4,6}
	%{
	%	\begin{scope}[start chain]
	%		\chainin (m0\i);
	%		\chainin (m0\j) [join=by dspconn];
	%		\chainin (m0\k) [join=by dspline];
	%		\chainin (m1\k) [join=by dspconn];
	%		\chainin (m2\k) [join=by dspconn];
	%	\end{scope}
	%	\draw[dspconn] (m2\i) -- (m2\k);
	%}

	%\draw[dspflow] (m06) -- (m09);
	\node at (-4.0,2.5) [] {\textbf{a)}};
	\node at (-4.0,-0.1) [] {\textbf{b)}};

\end{tikzpicture}

\end{document}
