\documentclass[14pt,border=10pt]{standalone}
\usepackage{tikz}
\usepackage[american]{circuitikz}
\usepackage{textcomp}

\usetikzlibrary{shapes,arrows}
\usetikzlibrary{dsp,chains}

\begin{document}
% Definition of blocks:
\tikzset{%
  block/.style    = {draw, thick, rectangle, rounded corners, fill={rgb:black,1;white,10},align=center, minimum height = 3em, minimum width = 3em},
  blockg/.style    = {draw, thick, rectangle, align=center, minimum height = 3cm, minimum width = 3cm},
  blockgg/.style    = {draw, thick, rectangle, align=center, minimum height = 3cm, minimum width = 2cm},
  mult/.style      = {draw, circle, node distance = 2cm}, % Adder
  input/.style    = {coordinate}, % Input
  output/.style   = {coordinate} % Output
}
% Defining string as labels of certain blocks.
\newcommand{\suma}{\Large$+$}
\newcommand{\mult}{\Huge x}
\newcommand{\inte}{$\displaystyle \int$}
\newcommand{\derv}{\huge$\frac{d}{dt}$}

\begin{tikzpicture}[auto, thick, node distance=2cm, >=triangle 45]

%\draw[help lines,step=0.5cm] (-1,-8) grid (24,20); 
				%\draw [rounded corners, name=bigfir] (4,20) rectangle (12,7.5);
	\draw [block,name=e00] (14,19.5) rectangle (16,18.5) node at (15,19){$\bar{p}_0(n)$};
	\draw [block,name=e01] (14,18.4) rectangle (16,17.4) node at(15,17.9){$\bar{p}_1(n)$};
	\draw [block,name=e02] (14,17.3) rectangle (16,16.3) node at(15,16.8){$\bar{p}_2(n)$};
	\draw node at (14.5,16) [scale=2]{$\vdots$};
	\draw node at (15.5,16) [scale=2]{$\vdots$};
	\draw node at (13.0,16) [scale=2]{$\vdots$};
	\draw node at (17.0,16) [scale=2]{$\vdots$};
	\draw [block,name=e03] (14,15) rectangle (16,14) node at (15,14.5){$\bar{p}_{M-1}(n)$};

	%ahora las entradas
	\draw node at (13,19)[name=ine00]{\Large \textopenbullet};
	\draw node at (13,17.9)[name=ine01]{\Large \textopenbullet};
	\draw node at (13,16.8)[name=ine02]{\Large \textopenbullet};
	\draw node at (13,14.5)[name=ine03]{\Large \textopenbullet};
	%conexiones de entrada
	\draw [->] (13.1,19) -- (14,19);
	\draw [->] (13.1,17.9) -- (14,17.9);
	\draw [->] (13.1,16.8) -- (14,16.8);
	\draw [->] (13.1,14.5) -- (14,14.5);
	%salidas
	\draw node at (16,19)[name=oute00]{};
	\draw node at (17,19)[name=oute]{\Large \textbullet};
	\draw node at (16,17.9)[name=oute01]{};
	\draw node at (17,17.9)[name=oute]{\Large \textbullet};
	\draw node at (16,16.8)[name=oute02]{};
	\draw node at (17,16.8)[name=oute]{\Large \textbullet};
	\draw node at (16,14.5)[name=oute03]{};
	%conexiones de salida
	\draw [->] (17,19) -- (18,19);
	\draw [->] (16,19) -- (17,19);
	\draw [->] (16,17.9) -- (17,17.9);
	\draw [->] (16,16.8) -- (17,16.8);
	\draw [->] (16,14.5) -- (17,14.5);
	%conmutador de entrada I
	\draw [->] (12.4,19.5) -- (12.4,18.3);
	\draw [->] (12,16.8) |- (12.9,19);
	\draw [-] (11,16.8) -- (12,16.8);

	\draw [->] (6,19) -- (7.5,19);
	%lineas de salida verticales 
	\draw [->] (17,17.7) -- (17,18.95);
	\draw [->] (17,16.7) -- (17,17.8);
	\draw [->] (17,16.2) -- (17,16.7);
	\draw [-] (17,14.5) -- (17,15.3);

	\node [above,scale=1] at (11,16.8) {$x(m)$};
	\node [above,scale=1] at (18,19) {$y(n)$};
	\node [above,scale=1] at (11,13.5) {$F_s$};
	\node [above,scale=1] at (13.3,13.5) {$F_s/M$};
	\node [above,scale=1] at (16.8,13.5) {$F_s/M$};

	%segundo arreglo de polifases
	\draw [block,name=e00] (3,19.5) rectangle (4,18.5) node	at(3.5,19){$\downsamplertext{M}$};
	\draw [block,name=e01] (3,18.4) rectangle (4,17.4) node at(3.5,17.9){$\downsamplertext{M}$};
	\draw [block,name=e02] (3,17.3) rectangle (4,16.3) node at(3.5,16.8){$\downsamplertext{M}$};
	\draw node at (3.5,16) [scale=2]{$\vdots$};
	\draw node at (2.0,16) [scale=2]{$\vdots$};
	\draw node at (4.5,16) [scale=2]{$\vdots$};
	\draw node at (9.0,16) [scale=2]{$\vdots$};
	\draw [block,name=e03] (3,15) rectangle (4,14) node at(3.5,14.5){$\downsamplertext{M}$};

	\draw [block,name=e00] (6,19.5) rectangle (8,18.5) node at(7,19){$\bar{p}_0(n)$};
	\draw [block,name=e01] (6,18.4) rectangle (8,17.4) node at(7,17.9){$\bar{p}_1(n)$};
	\draw [block,name=e02] (6,17.3) rectangle (8,16.3) node at(7,16.8){$\bar{p}_2(n)$};
	\draw node at (6.5,16) [scale=2]{$\vdots$};
	\draw node at (7.5,16) [scale=2]{$\vdots$};
	\draw [block,name=e03] (6,15) rectangle (8,14) node at(7,14.5){$\bar{p}_{M-1}(n)$};

	\draw node at (2,19)[name=ine10]{\Large \textbullet};
	\draw node at (2,17.9)[name=ine11]{\Large \textbullet};
	\draw node at (2,16.8)[name=ine12]{\Large \textbullet};
	\draw node at (2,14.5)[name=ine13]{\Large \textbullet};
	\draw [->] (2.1,19)   -- (3,19); 
	\draw [->] (2.1,17.9) -- (3,17.9);
	\draw [->] (2.1,16.8) -- (3,16.8);
	\draw [->] (2.1,14.5)  -- (3,14.5);
	\draw node at (9,19)[name=oute10]{};
	\draw node at (9,19)[name=oute]{\Large \textbullet};
	\draw node at (9,17.9)[name=oute11]{};
	\draw node at (9,17.9)[name=oute]{\Large \textbullet};
	\draw node at (9,16.8)[name=oute12]{};
	\draw node at (9,16.8)[name=oute]{\Large \textbullet};
	\draw node at (9,14.5)[name=oute13]{};
	\draw [->] (4,19)   -- (6,19); 
	\draw [->] (4,17.9) -- (6,17.9);
	\draw [->] (4,16.8) -- (6,16.8);
	\draw [->] (4,14.5)  -- (6,14.5);

	\draw [->] (8,19)   -- (9,19); 
	\draw [->] (8,17.9) -- (9,17.9);
	\draw [->] (8,16.8) -- (9,16.8);
	\draw [->] (8,14.5)  -- (9,14.5);
	%conmutador de entrada Q
	%\draw [->] (15.4,13.5) -- (15.4,18.3);
	%\draw [->] (15,16.8) |- (15.9,19);
	%\draw [-] (13,16.8) -- (15,16.8);
	%\draw [-] (16,15.8) -- (17,15.8);
	%conmutador de salida I
	\draw [->] (9,19) -- (10,19);
	%\draw [->] (7,19) |- (8,16.8);
	%conmutador de salida Q
	%\draw [->] (10.6,19.5) -- (10.6,18.3);
	\draw [-] (0,19) -- (1,19);
	\draw [->] (1,19) -- (1.9,19);
	%lineas verticales
	\draw [<-] (2,18) -- (2,18.95);
	\draw [<-] (2,16.9) -- (2,17.8);
	\draw [<-] (2,16.2) -- (2,16.7);
	\draw [<-] (2,14.6) -- (2,15.4);

	\draw [->] (9,18) -- (9,18.95);
	\draw [->] (9,16.9) -- (9,17.8);
	\draw [->] (9,16.2) -- (9,16.7);
	\draw [->] (9,14.5) -- (9,15.4);
	%labels
	\node [above,scale=1] at (10,19.0) {$y(n)$};
	\node [above,scale=1] at (0,19.0) {$x(m)$};
	\node [left,scale=1] at (2,18.5) {$z^{+1}$};
	\node [left,scale=1] at (2,17.5) {$z^{+1}$};
	\node [left,scale=1] at (2,15.0) {$z^{+1}$};

	\node [left,scale=1.3] at (5,13.0) {\textbf{a)}};
	\node [left,scale=1.3] at (15.3,13.0) {\textbf{b)}};

\end{tikzpicture}
\end{document}
