\ifcase0  % choose 0=slides, 1=article, 2=refart
	 \documentclass[ignorenonframetext,12pt]{beamer}
	 \geometry{paper=a6paper,landscape}
\or\documentclass[a4paper,11pt]{article}
	 \usepackage{url,beamerarticle}
\or\documentclass[a4paper,11pt]{refart}
	 \let\example\relax
	 \usepackage{url,beamerarticle}
\fi

\ifcase0  % choose a theme like these
%\usetheme{Montpellier}% I recommend
	 \usetheme{default}% I recommend
\or\usetheme{Singapore}
\or\usetheme{Szeged}
\or\usetheme{Boadilla}
\or\usetheme{Pittsburgh}
\or\usetheme{Madrid}
\or\usetheme{Warsaw} % common choice, but often poor
\fi

\usepackage[utf8]{inputenc}%para acentos en español
\usepackage{graphicx,pgfplots,parskip}
\usepackage{siunitx}
\graphicspath{{media/}}
\usepackage{xcolor}
\usepackage[percent]{overpic}

\title{Microelectrónica y técnicas avanzadas de procesamiento digital con
aplicaciones en micro-resonadores superconductores multipíxeles}
\subtitle{\alert{Charla de avance}}
\author{Ing. L. Horacio Arnaldi\\
\vspace{0.4cm}
Director: Dr. Ing. Damián Dellavale\\
Co-Director: Dr. Ing. José Lipovetzky\\
\vspace{0.6cm}
Instituto Balseiro}
\date{12 de Diciembre, 2019}

\begin{document}

\begin{frame}
				\maketitle
\end{frame}

\begin{abstract}
				This abstract, being outside the frame environment, does not appear in
				the presentation.  Your outline will be the basis for a couple of
				sentences of talk for each of the following questions:
				\begin{itemize}
								\item What was done?
								\item Why do it?
								\item What were the results?
								\item What do the results mean in theory and/or practise?
								\item What is the reader's benefit?
								\item How can the readers use this information for themselves? 
				\end{itemize}
\end{abstract}

%\begin{frame}{Outline}
%				\tableofcontents
%\end{frame}

%------------------------------------------------------------------------------
\section{Motivación}
\begin{frame}{Motivación}
				\begin{itemize}
								\item Aplicación (CMB, Dark Mater)
								\item Detectores (MKID, QUBIC)
								\item Electrónica (FPGA, GPU, CPU, etc)
				\end{itemize}
												\only<1>{\includegraphics[width=0.45\textwidth]{motivacion1}}
												\only<1>{\includegraphics[width=0.45\textwidth]{motivacion4}}
												\only<2>{\includegraphics[width=0.3\textwidth]{qubic1}}
												\only<2>{\includegraphics[width=0.6\textwidth]{qubic2}}
												\only<3>{\includegraphics[width=0.7\textwidth]{zcu2}}
				
\end{frame}

%\begin{frame}{Motivación}
%        \begin{itemize}
%                \item ¿Qué me motiva a hacer esto?
%                \item Desarrollar un grupo de estudios/mediciones en RF a bajas
%                        temperaturas
%                \item Desarrollo de nuevas herramientas/instrumentos potenciados
%                        por el uso de dispositivos lógicos programables (FPGA,
%                        GPU, CPU, etc.) de última generación
%                \item ¿Por qué lo hago?
%                \item En los últimos años se viene trabajando fuertemente en
%                        detectores de baja temperatura y altas frecuencias
%                \item ¿Qué beneficios obtengo/obtenemos?
%                \item Creación de grupos inter-disciplinarios para el estudio de
%                        fenómenos poco conocidos/estudiados en el CAB
%                \item ¿Qué grupos integro/se formaron con esto?
%                \item Grupo de BT, Nano-Microelectrónica, PyC
%                \item
%        \end{itemize}
%\end{frame}


\begin{frame}{Contexto}
				\framesubtitle{Estudios del CMB}
												\only<1>{\includegraphics[width=0.5\textwidth]{c1_cmb_map}}
												%\only<1>{\includegraphics[width=0.4\textwidth]{motivacion1}}
												\only<1>{\includegraphics[width=0.4\textwidth]{motivacion3}}
												%\only<2>{\includegraphics[width=0.4\textwidth]{motivacion3}}
				\begin{itemize}
												\begin{columns}
																\begin{column}{0.5\textwidth}
																\item[*] {\color{blue}Radiación de fondo de microondas (CMB)}
																				$\to$ radiación EM	de una etapa temprana del universo (Big	Bang)
																\item[*] Penzias y Wilson (1964). TELSTAR (1962) 
																\item[*] Ajuste a la curva teórica para un \alert{cuerpo negro
																				de 2.728 K}
																\end{column}
																\begin{column}{0.5\textwidth}
																				\includegraphics[width=0.9\textwidth]{motivacion2}
																\end{column}
												\end{columns}
				\end{itemize}

\end{frame}
%\begin{frame}{Motivación}
%				\framesubtitle{Estudios del CMB}
%				\begin{columns}
%								\begin{column}{0.49\textwidth}
%				\begin{itemize}
%								\item[*] El hecho de que el CMB tenga un espectro de cuerpo
%												negro tan preciso es evidencia de que vino de cuando era
%												mucho más caliente y denso de lo que es ahora
%								\item[*] Por lo tanto, el espectro del CMB es una fuerte
%												evidencia de que el Universo experimentó una etapa de
%												``Big Bang caliente''
%				\end{itemize}
%								\end{column}
%								\begin{column}{0.49\textwidth}
%												\only<1>{\includegraphics[width=0.9\textwidth]{motivacion5}}
%												\only<1>{\includegraphics[width=0.9\textwidth]{c1_cmb_map}}
%								\end{column}
%								\end{columns}
%\end{frame}
\begin{frame}{Motivación}
				\framesubtitle{Estudios del CMB}
								Expansión del Universo $\to$ \alert{temperatura baja}
				\begin{columns}
								\begin{column}{0.4\textwidth}
												\begin{itemize}
																				%\item[*] Expansión del Universo $\to$ \alert{temperatura baja}
																\item[*] \alert{Anisotropías} $\to$ colores indican variaciones de
																				temperatura: rojo para más caliente, azul para más
																				frío
																				\includegraphics[width=0.9\textwidth]{motivacion5}
																				%\item[*] Tamaños de los puntos calientes y fríos $\to$
																				%				cálculos de valores fundamentales para la forma, tamaño,
																				%				edad, contenido, tasa de expansión (y más) de nuestro
																				%				universo
												\end{itemize}
								\end{column}
								\begin{column}{0.59\textwidth}
												\includegraphics[width=0.9\textwidth]{c1_cmb_map}
												\vspace{1cm}

												Tamaños de los puntos calientes y fríos $\to$
												cálculos de valores fundamentales para la forma, tamaño,
												edad, contenido, tasa de expansión (y más) de nuestro
												universo
								\end{column}
				\end{columns}
\end{frame}
%\begin{frame}{Motivación}
%				\framesubtitle{QUBIC}
%												\only<1>{\includegraphics[width=0.4\textwidth]{c1_cmb_map}}
%												\only<1>{\includegraphics[width=0.4\textwidth]{motivacion1}}
%												\only<2>{\includegraphics[width=0.4\textwidth]{motivacion2}}
%												\only<2>{\includegraphics[width=0.4\textwidth]{motivacion3}}
%												\only<3>{\includegraphics[width=0.6\textwidth]{motivacion4}}
%				\begin{itemize}
%								\item[*] 
%												
%								\item[*] 
%								\item[*] 
%								\only<3>{\item[*] Sus datos (y muchos otros desde entonces) se ajustan a
%												la curva teórica para un cuerpo negro de 2.728 K}
%				\end{itemize}
%\end{frame}

%------------------------------------------------------------------------------
\section{MKIDs}
\begin{frame}{\textbf{M}icrowave \textbf{K}inetic \textbf{I}nductance
				\textbf{D}etector (MKID)}
				\centering
												\qquad \includegraphics[width=0.8\textwidth]{concepto_mkid1}
				\begin{itemize}
								\item \footnotesize{Superconductores $\to$ inductancia AC debida a la
												inercia de los pares de Cooper}
												%\begin{itemize}
												%				\item[*] \scriptsize{{\color{blue}alternativamente, debido a la
												%								energía almacenada en la supercorriente
												%								de apantallamiento}}
												%\end{itemize}
								\item Cambia cuando los pares de Cooper se rompen debido a la
												energía de entrada
								\item Se mide el cambio monitoreando un circuito resonante
								\item Punto crítico $\to$ los \alert{superconductores proveen un muy
												alto Q} ($Q_i > 10^7$) $\to$ miles de resonadores
												pueden ser leídos a través de una sola linea 
												\begin{itemize}
																\item[*] \scriptsize{{\color{blue}componentes de lectura criogénica muy
																				simples}}
												\end{itemize}
				\end{itemize}

\end{frame}
\begin{frame}{\textbf{M}icrowave \textbf{K}inetic \textbf{I}nductance
				\textbf{D}etector (MKID)}
				\begin{columns}
								\begin{column}{0.49\textwidth}
												\begin{equation*}
																Z_s = R_s + j \omega L_s
												\end{equation*}
												\begin{equation*}
																L_s = L_g + L_k
												\end{equation*}

												\flushleft	\includegraphics[width=0.6\textwidth]{LCR_mkid}
								\end{column}
								\begin{column}{0.59\textwidth}
												\begin{itemize}
																\item[o] $R_s \to$ pérdidas A.C. debidas a $e^-$ no apareados
																				(pares de Cooper)
																\item[o] $L_s \to$ inductancia superficial total
																\item[o] $L_g \to$ inductancia geométrica
																\item[o] $L_k \to$ inductancia cinética
																				({\color{blue}dependiente de T}), además
												\end{itemize}

												\begin{equation*}
																L_k \propto 1/T_c
												\end{equation*}

												\begin{equation*}
																f_r = \frac{1}{\sqrt{C_g(L_g + L_k)}}
												\end{equation*}
								\end{column}
				\end{columns}
\end{frame}

%\begin{frame}{MKIDs}
%				\begin{columns}
%								\begin{column}{0.49\textwidth}
%												\centering
%												\includegraphics[width=0.92\textwidth]{pulso_respuesta_mkid2}
%								\end{column}
%								\begin{column}{0.59\textwidth}
%												\footnotesize{Mecanismo de detección de los KIDs $\to$ un
%												cambio temporal en la inductancia cinética superficial
%												de un superconductor cuando se absorbe un fotón de
%												energía $h\nu \geq 2 \Delta(T)$, donde $\Delta(T)$ es el
%												parámetro de gap superconductor (teoría BCS).
%												\pause
%
%												Los fotones de alta energía rompen los pares de Cooper.
%												Estos se recombinan en un tiempo característico, del
%												orden de $10-500\,\mu\text{s}$,	{\color{red}produciendo
%												cambios similares a	pulsos en la impedancia	
%												superficial.}
%												\pause
%
%												Como la impedancia del superconductor es principalmente
%												inductiva, especialmente para $T << T_c$,
%												el superconductor puede diseñarse como el elemento
%												inductivo en un circuito resonante tipo RLC.
%												\pause
%
%												{\color{blue}El factor de calidad (Q) del circuito
%												resonante determina ambos, la \alert{sensibilidad} y la
%												\alert{velocidad} de respuesta del dispositivo}.}
%								\end{column}
%				\end{columns}
%\end{frame}
%\begin{frame}{MKIDs}
%				\framesubtitle{Resolución de Energía}
%				%\begin{exampleblock}{}
%				%       Works on the principle that incident photons change the surface
%				%       impedance of a superconductor through the \textit{kinetic
%				%       inductance effect}
%				%\end{exampleblock}
%												\begin{equation*}
%																R = \frac{1}{2.355}\sqrt{\frac{\eta h \nu}{F
%																\Delta}}
%												\end{equation*}
%				$\eta = 0.57 \to$ eficiencia para crear cuasipartículas (típico),
%
%				$h\nu \to$ energía del fotón incidente,
%
%				$\Delta = 1.72\,k_BT_c \to$ gap de energía del absorvedor superconductor,
%
%				$F \approx 0.2 \to$ factor de Fano
%
%				$R=150$ a 5\,eV para temperatura de operación de 100\,mK.
%
%				$T = 15$\,mK $\to$ resolución de energía máxima teórica de \alert{$R = 400$ a
%				5\,eV} (aunque es probable que otras fuentes de ruido, como el ruido de
%				dos niveles del sistema (TLS), sean más importantes a medida que el
%				desarrollo futuro aumente	la resolución de energía).
%
%\end{frame}
%\begin{frame}{MKIDs}
%				\scriptsize{\begin{itemize}
%								\item Como detector de fotones en el rango mm/submm
%								\item Resonancia  $1-10\,\text{GHz}$
%								\item $Q \sim 10^5$
%								\item $T_{op} < 300\,\text{mK}$
%								\item Resolución de energía intrínseca $R = \frac{E}{\Delta E} \sim
%												20-150$\footnote{Rev.Sci.Instr. 83, 044702 (2012)}
%								\item Resonadores diseñados para operar separados por 2\,MHz en
%												un BW de $4-5\,\text{GHz}$.
%								\item Transmisión fuera de resonancia $\simeq 1$
%				\end{itemize}
%				Cada resonador tendrá un $BW \sim 200\,\text{kHz}$ (\alert{basado en el
%				factor de calidad que necesitamos}), se requiere una separación de
%				2\,MHz entre los resonadores (la posición del resonador se moverá en
%				función de una carga diferente. Si los empaquetamos demasiado cerca uno
%				del otro, su posición relativa puede cambiar durante la observación o
%				en diferentes ciclos de refrigerado).
%
%				Por lo tanto, queremos que ADC tenga BW de al menos 400\,MHz y una SNR
%				superior a 59\,dB.
%
%				Si solo consideramos el ruido de cuantización, teóricamente, una SNR de
%				59\,dB requerirá que el ADC tenga al menos 10 bits.
%				}
%				\tiny{\textbf{arXiv:1310.5891v2 (2013)}}
%\end{frame}
%\begin{frame}{Tipos de MKIDs}
%				\begin{itemize}
%								\item OLE Doyle et.al 2008
%								\item ¿Por qué lo hago?
%								\item ¿Qué beneficios obtengo/obtenemos?
%								\item ¿Qué grupos integro/se formaron con esto?
%								\item 
%								\item 
%				\end{itemize}
%
%\end{frame}

%------------------------------------------------------------------------------
%\section{Aplicaciones}
%\begin{frame}{Polarización del CMB}
%				\begin{itemize}
%								\item OLE Doyle et.al 2008
%								\item ¿Por qué lo hago?
%								\item ¿Qué beneficios obtengo/obtenemos?
%								\item ¿Qué grupos integro/se formaron con esto?
%								\item 
%								\item 
%				\end{itemize}
%
%\end{frame}
%
%\begin{frame}{Antenna-Coupled multicolor MKIDs}
%				\begin{itemize}
%								\item OLE Doyle et.al 2008
%								\item ¿Por qué lo hago?
%								\item ¿Qué beneficios obtengo/obtenemos?
%								\item ¿Qué grupos integro/se formaron con esto?
%								\item 
%								\item 
%				\end{itemize}
%
%\end{frame}
%
%\begin{frame}{Detección de fonones usando MKIDs}
%				\begin{itemize}
%								\item OLE Doyle et.al 2008
%								\item ¿Por qué lo hago?
%								\item ¿Qué beneficios obtengo/obtenemos?
%								\item ¿Qué grupos integro/se formaron con esto?
%								\item 
%								\item 
%				\end{itemize}
%
%\end{frame}

%------------------------------------------------------------------------------
\section{Mediciones}
%\subsection{Diseño FNAL}%es el diseño de Israel
%\begin{frame}{Barrido en potencia}
%				\centering
%												\includegraphics[width=0.4\textwidth]{isra_mkid1}
%												\includegraphics[width=0.32\textwidth]{isra_mkid2}
%\end{frame}
%\begin{frame}{Barrido en potencia}
%				\centering
%												\includegraphics[width=0.4\textwidth]{isra_mkid3}
%												\includegraphics[width=0.32\textwidth]{isra_mkid4}
%\end{frame}
%\begin{frame}{Barrido en potencia}
%				\centering
%												\includegraphics[width=0.4\textwidth]{isra_mkid1}
%												\includegraphics[width=0.32\textwidth]{isra_mkid2}
%\end{frame}
\subsection{Diseño UCSB}
\begin{frame}{Diseño de la Universidad de California, Santa Bárbara (UCSB)}
				%\begin{frame}{Diseño UCSB} %Estudio de la $T_c$}
				\framesubtitle{Características}
				\begin{columns}
								\begin{column}{0.49\textwidth}
												\begin{itemize}
																%\item[o] Diseño de la UCSB
																\item[o] Arreglo utilizado en
																				{\color{blue}ARCONS} (Array
																				Camera for Optical to Near-IR
																				Spectrophotometry)
																\item[o] Arreglo de 44 x 46 resonadores (\alert{2024
																				píxeles})
																\item[o] Separación entre frecuencias de
																				resonancia $\to$ \alert{2 MHz}
																\item[o] Nitruro de Titanio (TiN) con una temperatura
																				crítica de \alert{$\sim$ 0.9 K}
																\item[o] Banda de observación: \alert{380 nm --
																				1150 nm}
												\end{itemize}
								\end{column}
								\begin{column}{0.49\textwidth}
												\centering
												\includegraphics[width=0.6\textwidth]{mkid4}
												\includegraphics[angle=-90,width=0.62\textwidth]{mkid2}
								\end{column}
				\end{columns}
\end{frame}
\begin{frame}{Diseño de la Universidad de California, Santa Bárbara (UCSB)}
%\begin{frame}{Estudio de la $T_c$}
				\framesubtitle{Mediciones preliminares para definir los requerimientos
				de la electrónica}
				\centering
												\includegraphics[width=0.61\textwidth]{mkid_ucsb2}
				%\begin{columns}
				%				\begin{column}{0.49\textwidth}
												\footnotesize{\begin{itemize}
																\item[o] TiN sobre Si
																\item[o] TiN de 60 nm
																\item[o] Wire bonds de oro a los pads para
																				disipación de calor
																\item[o] Caja de cobre enchapada en oro (solo
																				materiales no magnéticos)
																%\item[o] Wire bonds de Al a las líneas
																%				de Tx

																				%	Las pestañas sostienen el chip hacia
																				%	abajo y ayudan a aplastar el pegamento
																				%	para que el chip quede al ras del brazo.
												\end{itemize}}
				%				\end{column}
				%				\begin{column}{0.49\textwidth}
				%								\includegraphics[width=0.72\textwidth]{mkid_ucsb2}
				%								%\centering
				%								%%\only<1>{\includegraphics[width=0.5\textwidth]{mkid4}}
				%								%\only<1>{\includegraphics[angle=-90,width=0.72\textwidth]{mkid2}}
				%								%\only<2>{\includegraphics[width=0.82\textwidth]{mkid5}}
				%								%\only<2>{\includegraphics[width=0.82\textwidth]{mkid6}}
				%								%\only<3>{\includegraphics[width=0.82\textwidth]{mkid7}}
				%								%\only<3>{\includegraphics[width=0.82\textwidth]{mkid8}}
				%				\end{column}
				%\end{columns}
\end{frame}
%\begin{frame}{Diseño de la Universidad de California, Santa Bárbara (UCSB)}
%				\framesubtitle{Características}
%				\centering
%				%				\includegraphics[width=0.42\textwidth]{resonador0_asc_sin_filtro}
%												\includegraphics[width=0.62\textwidth]{mkid_ucsb1}
%\end{frame}
%\begin{frame}{Diseño de la Universidad de California, Santa Bárbara (UCSB)}
%				\framesubtitle{Características (Charla de Donna Kubik, FNAL)}
%
%				\centering
%				%				\includegraphics[width=0.42\textwidth]{resonador0_asc_sin_filtro}
%												\includegraphics[width=0.72\textwidth]{mkid_ucsb2}
%\end{frame}
\begin{frame}{Estudio del cambio fraccional de frecuencia de resonancia}
				\framesubtitle{Mediciones preliminares para definir los requerimientos
				de la electrónica}
				Atomic Layer Deposition (ALD)
												\begin{equation*}
																\frac{\delta f_r}{f_r} = \frac{f_r(T) -
																f_r(0)}{f_r(0)} = -\frac{1}{2}\alpha\frac{\delta
																\sigma_2(T)}{\sigma_2(0)}
												\end{equation*}
				\begin{columns}
								\begin{column}{0.49\textwidth}
												{\color{blue}\begin{equation*}
								\frac{\delta f(T)}{f(0)} = 
																-\frac{\alpha}{2}\sqrt{\frac{\pi \Delta}{2 k_B
																T}} e^{-\frac{\Delta}{k_B T}}
												\end{equation*}}
				\begin{equation*}
								\Delta = \frac{3.5}{2} k_B T_c
				\end{equation*}
				\begin{equation*}
								\alpha = \frac{L_k}{L_\text{tot}}
				\end{equation*}

												$\Delta \to$ gap de energía
								\end{column}
				\begin{column}{0.49\textwidth}
				\begin{equation*}
								\sigma = \sigma_1 - j \sigma_2
				\end{equation*}

								En películas de TiN, $\sigma$, se desvía cada vez más de la
								teoría convencional	de Mattis-Bardeen debido a un \alert{fuerte
								desorden de TiN}
								\end{column}
								\end{columns}
\end{frame}
\begin{frame}{Estudio de la $T_c$}
				\framesubtitle{Mediciones preliminares para definir los requerimientos
				de la electrónica}
				%\only<1>{\begin{equation*}
				%				\frac{\sigma_2(T)}{\sigma_N} \simeq
				%												\frac{\pi \Delta}{\hbar
				%												\omega}\left[1-2\exp \left(-\frac{\Delta}{k_B
				%												T}\right)\exp \left(-\frac{\hbar \omega}{2 
				%												k_B T}\right)
				%												I_0 \left(\frac{\hbar \omega}{2 k_B T}\right)\right]
				%\end{equation*}}
				Resultados similares a los medidos por la UCSB para otros arreglos
				%\vspace{1cm}
				\begin{columns}
								\begin{column}{0.69\textwidth}
				\centering
												%\only<1>{\includegraphics[width=0.82\textwidth]{delta_f_vs_temp}}
												\includegraphics[width=0.54\textwidth]{medicion_Tc_ucsb}
								\end{column}
				\begin{column}{0.39\textwidth}
												{\color{blue}\begin{equation*}
								\frac{\delta f(T)}{f(0)} = 
																-\frac{\alpha}{2}\sqrt{\frac{\pi \Delta}{2 k_B
																T}} e^{-\frac{\Delta}{k_B T}}
												\end{equation*}}
								\end{column}
								\end{columns}
												\includegraphics[width=0.75\textwidth]{delta_f_vs_temp}
\end{frame}
%\begin{frame}{Barrido en potencia}
%				\begin{itemize}
%								\item[*] -25 dB hasta -50 dB + atenuador de 30 dB
%				\end{itemize}
%				\centering
%												\includegraphics[width=0.48\textwidth]{resonador0_asc_sin_filtro}
%												\includegraphics[width=0.48\textwidth]{resonador0_asc_filtro}
%\end{frame}
%\begin{frame}{Barrido en potencia}
%				\centering
%				%				\includegraphics[width=0.42\textwidth]{resonador0_asc_sin_filtro}
%												\includegraphics[width=0.82\textwidth]{resonador0_asc_filtro}
%\end{frame}
%\begin{frame}{El amplificador criogénico}
%				%\begin{itemize}
%				%\item 
%				El ruido de fase de doble banda lateral (DSB) del amplificador HEMT
%												viene dado por la expresión
%												{\color{red}
%												\begin{equation*}
%																N_{\phi \text{DSB}} = \frac{k_B T_n}{P}
%												\end{equation*}} 
%												\flushleft
%												$T_n \to$ temperatura de ruido del amplificador, 
%
%												$P \to$ potencia en la entrada. 
%
%												La potencia de lectura de cada MKID debería ser de $-100 \pm
%												15\,\text{dBm}$ ({\color{red} a confirmar}). 
%
%												Para una potencia de lectura de $-85\,\text{dBm}$, fuera
%												de resonancia, y $T_n = 6\,\text{K}$, se espera
%												$-106\,\text{dBc/Hz}$ para el ruido de fase del HEMT.
%												Idealmente, la electrónica de lectura contribuirá con
%												ruido muy por debajo de este
%												valor.\footnote{Rev.Sci.Instr. 83, 044702 (2012)}
%												%\end{itemize}
%\end{frame}

\begin{frame}{$N_\text{canales}$ vs. $P_\text{lectura}$}
				\footnotesize{\textbf{Rev.Sci.Instr. 83, 044702 (2012)}}
				\begin{columns}
								\begin{column}{0.4\textwidth}
												\begin{itemize}
																\item \footnotesize{Dada la especificación de
																				voltaje de ruido del ADC, se puede
																				calcular el número de canales
																				{\color{red}$n$} que el ADC puede leer
																				a una potencia dada sin aumentar el
																				ruido del HEMT (amplificador)}
																\item Los contornos de la figura se calculan
																				escalando  el ruido del HEMT con un
																				factor dependiente del canal, $k_B
																				T_n /(p_\text{max}^2 P)$.
																\item $p_\text{max} \propto
																				n^{1/2}\quad\text{para}\quad n >> 1$
																\item {\color{blue}Se asume fase aleatoria por
																				tono (factor de cresta)}
												\end{itemize}
								\end{column}
								\begin{column}{0.6\textwidth}
												\centering
												\includegraphics[width=0.8\textwidth]{power_vs_Nchannels2}
												\includegraphics[width=0.9\textwidth]{in_spectrum_time}
								\end{column}
				\end{columns}
\end{frame}

\begin{frame}{Barrido en potencia}
				\framesubtitle{Mediciones preliminares para definir los requerimientos
				de la electrónica}
				\footnotesize{\begin{itemize}
								\item[*] Configuración: -25 dB hasta -50 dB + atenuador de 30 dB
												(externo)
								\item[*] Ruido vs. Potencia
								\item[*] Filtro moving average (200 puntos)
				\end{itemize}}
				%\vspace{-1cm}
				\centering
												\includegraphics[width=.9\textwidth]{res0_asc_full_potencias}
\end{frame}
\begin{frame}{Potencia óptima de trabajo}
				\framesubtitle{Mediciones preliminares para definir los requerimientos
				de la electrónica}
				\footnotesize{\begin{itemize}
								\item[*] Potencia efectiva en los resonadores $\to$ -55 dB a
												-80 dB 
								\item[*] Efecto saturación para $P > -65\,\text{dB}$
								\item[*] \alert{$-77\,\text{dB} < P_\text{opt} < -65\,\text{dB}$}

				\end{itemize}}
				\centering
												\includegraphics[width=0.89\textwidth]{res0_3_potencias}

\end{frame}
%\begin{frame}{Barrido en potencia}
%				\centering
%												\includegraphics[width=0.82\textwidth]{res0_Q_vs_P}
%
%\end{frame}
\begin{frame}{Q vs Potencia}
				\framesubtitle{Mediciones preliminares para definir los requerimientos
				de la electrónica}
				\begin{columns}
								\begin{column}{0.35\textwidth}
				\footnotesize{\begin{itemize}
								\item[*] No existe efecto ``histéresis'' 
								\item[*] Saturación para $P > -65\,\text{dB}$
								%\item[*] \alert{$-77\,\text{dB} < P_\text{opt} <
								%				-65\,\text{dB}$}
				\end{itemize}}
								\end{column}
								\begin{column}{0.65\textwidth}
												\centering
												\includegraphics[width=0.45\textwidth]{un_resonador0_des_filtro}
												\includegraphics[width=0.45\textwidth]{un_resonador1_des_filtro}
								\end{column}
				\end{columns}

				\begin{columns}
								\begin{column}{0.49\textwidth}
												\qquad \qquad-55 dB $\to$ -80 dB
												\centering
												\includegraphics[height=0.45\textheight,width=1.1\textwidth]{Q_vs_P_tot_des}
								\end{column}
								\begin{column}{0.49\textwidth}
												-80 dB $\to$ -55 dB
												\centering
												\includegraphics[height=0.45\textheight,width=1.1\textwidth]{Q_vs_P_tot_asc}
								\end{column}
				\end{columns}

\end{frame}

%------------------------------------------------------------------------------
\section{Desarrollo del sistema de lectura}
\begin{frame}{Desarrollo del sistema de excitación/lectura}
				\framesubtitle{Temas a considerar}
				\begin{columns}
								\begin{column}{0.5\textwidth}
				\normalsize{\begin{itemize}
								\item Algoritmo de procesamiento de señales
								\item Selección de frecuencia, velocidad de datos de salida
								\item Ruido
								\item Potencia
								\item Rango dinámico
								\item $f$ espúreas, productos de intermodulación, etc.
								\item Implementación: ubicación, empaque, fuente de
												energía, comunicación, interfaz con PC, etc.
				\end{itemize}}
								\end{column}
								\begin{column}{0.5\textwidth}
												ZCU111/RedPitaya
				\includegraphics[width=0.8\textwidth]{zcu2}
				\includegraphics[width=0.8\textwidth]{zcu1}
												\includegraphics[width=0.8\textwidth]{600px-RedPitaya_HW_overview}
								\end{column}
								\end{columns}
\end{frame}
\begin{frame}{Desarrollo del sistema de excitación/lectura}
				\framesubtitle{Temas a considerar}
				\centering
				\includegraphics[width=\textwidth]{readout1}

				\begin{itemize}
								\item Ruido \alert{dominado por el amplificador criogénico},
												$T_\text{ruido} \sim$ 2 a 5\,K 
								\item \alert{El chip ADC será el próximo factor
												limitante para el ruido de lectura}
								\item	$f_s$ se elige para que
												coincida con el ancho de banda del resonador
				\end{itemize}

				%\tiny{\emph{\textbf{An open-source readout for MKIDs}}, Proc. SPIE Astron.
				%Telesc.  Instrum., doi:10.1117/12.856832}
\end{frame}

%\begin{frame}{Diseño y desarrollo del sistema de lectura}
%				\only<1>{\footnotesize{\begin{itemize}
%								\item Para leer un MKID, se pasa un tono de prueba a su
%												frecuencia resonante
%								\item Cuando el detector absorbe un fotón, la fase del tono
%												cambia repentinamente
%								\item A partir de este cambio, se determina la \alert{energía} y
%												el \alert{tiempo de llegada} del fotón.
%								\item En la lectura, los DAC generan un peine de frecuencia
%												complejo que se mezcla con el rango de frecuencias
%												resonantes
%								\item La señal resultante se envía a través de la matriz MKID,
%												se mezcla de nuevo a la banda base y se digitaliza
%												mediante un ADC
%								\item Esto es procesado por el firmware en la FPGA, que detecta
%												eventos de fotones en la fase del tono de prueba de cada
%												píxel
%								\item Los datos de los fotones se envían a una computadora de
%												adquisición y control de datos para su registro
%				\end{itemize}}}
%				\centering
%				\only<2>{\includegraphics[width=0.8\textwidth]{arcons_readout}}
%\end{frame}

%\begin{frame}{Firmware}
%				The firmware is written in VHDL/Verilog. 
%
%				It controls the DAC output, and processes the ADC input. 
%
%				It separates the digitized readout signal into frequency bins (channels)
%				corresponding to each pixel. 
%
%				The signal for each pixel is filtered and converted to phase before
%				being run through \alert{pulse detection}.
%
%				The output data rate is expected to be around 100 complex $S_{21}$
%				measurements per second, per resonator\footnote{\tiny{from
%				MKIDCam Readout Electronics Specifications (2008)}} $\to$ (\alert{to be
%				confirmed})
%\end{frame}
%
%\begin{frame}{Pulse detection}
%				\begin{columns}
%								\begin{column}{0.49\textwidth}
%												\scriptsize{Readout capable of recording to file one continuous
%												second of phase data for one pixel at a time, \alert{sampled at
%												1\,MHz}. 
%
%												To record data from all the pixels, though, the firmware must
%												reduce the data to only include data from detected photons,
%												which are seen as negative peaks in the phase.}
%								\end{column}
%								\begin{column}{0.49\textwidth}
%												\centering
%												\includegraphics[width=0.98\textwidth]{pulsos_de_fase}
%								\end{column}
%				\end{columns}
%				\scriptsize{A (programmable) filter is applied to the phase signal to increase the
%				SNR of drops in phase due to photons.  
%
%				{\color{blue}Snapshots	of raw phase data are used to create a
%				\alert{matched filter},	customized to each pixel.}
%
%				The {\color{blue}firmware} also keeps track of the phase baseline and subtracts it
%				before checking for photon pulses, to resist false triggers from $1/f$
%				noise.\footnote{ARCONS talks.}}
%\end{frame}
%\begin{frame}{Readout design and development}
%
%				MKIDs $\to$ 
%
%				The obvious advantage of MKIDs over competing cryogenic technologies
%				like TESs is the elimination of the cryogenic electronics required for
%				multiplexing
%
%				\tiny{\emph{\textbf{A readout for large arrays of microwave kinetic
%				inductance detectors}}, Rev. Sci. Instrum., doi:10.1063/1.3700812}
%\end{frame}

%\begin{frame}{Design requirements}
%				\begin{itemize}
%								\item The readout must not introduce significant noise above the
%												system noise floor set by the \textbf{cryogenic
%												amplifier} with a noise temperature of $\sim
%												4\,\text{K}$.
%								\item The entire readout system must be capable of
%												reading out at least \alert{512 resonators} in $\sim
%												1\,\text{GHz}$ of bandwidth ({\color{red} to be
%												confirmed}).
%								\item \alert{Crosstalk} between channels greater than
%												$250\,\text{kHz}$ apart should be less than \alert{1\%}.
%								\item Intrinsic energy resolution \alert{$R \sim 20-150$}
%												\footnote{Rev.Sci.Instr. 83, 044702 (2012)}
%				\end{itemize}
%\end{frame}

%------------------------------------------------------------------------------
%\section{ARCONS}
%\begin{frame}{ARCONS}
%
%				\textbf{A}rray \textbf{C}amera for \textbf{O}ptical to \textbf{N}ear-IR
%				\textbf{S}pectrophotometry
%				\begin{itemize}
%								\item Primer instrumento terrestre basado en MKIDs para el rango de
%												longitud de onda óptica hasta el IR cercano
%								\item Diseño óptico simple que permite muy alto rendimiento
%								\item Resolución de tiempo de hasta seis órdenes de magnitud
%												mejor que un CCD
%								\item Ancho de banda intrínseco extremadamente amplio (0.1
%												-5\,$\mu$m) con buena eficiencia cuántica
%								\item Sin ruido de lectura o corriente oscura, y casi perfecto
%												rechazo de rayos cósmicos
%								\item No se pierde tiempo de observación al leer la matriz
%				\end{itemize}
%
%				%\begin{center}
%				%				\includegraphics[width=0.8\textwidth]{FDM_channel_diagram}
%				%\end{center}
%\end{frame}
%\begin{frame}{ARCONS}
%
%				\textbf{A}rray \textbf{C}amera for \textbf{O}ptical to \textbf{N}ear-IR
%				\textbf{S}pectrophotometry
%				\begin{itemize}
%								\item La tecnología actual óptica MKID tiene una resolución
%												espectral $R = \lambda/\Delta \lambda \sim 10$ a
%												\SI{4000}{\angstrom}
%								\item 
%								\item Resolución de tiempo de hasta seis órdenes de magnitud
%												mejor que un CCD
%								\item Ancho de banda intrínseco extremadamente amplio (0.1
%												-5\,$\mu$m) con buena eficiencia cuántica
%								\item Sin ruido de lectura o corriente oscura, y casi perfecto
%												rechazo de rayos cósmicos
%								\item No se pierde tiempo de observación al leer la matriz
%				\end{itemize}
%
%				%\begin{center}
%				%				\includegraphics[width=0.8\textwidth]{FDM_channel_diagram}
%				%\end{center}
%\end{frame}

%------------------------------------------------------------------------------
\begin{frame}{Desarrollo del sistema de excitación/lectura}
\framesubtitle{Frequency Division Multiplexing (FDM)}
				\begin{columns}
								\begin{column}{0.5\textwidth}
												\begin{itemize}
																\item[o]	Estrategia de multiplexación poderosa\\
																				--\footnotesize{{\color{blue}Muchos detectores 
																				acoplados a una sola linea}}
																\normalsize{\item[o] Muchos canales por amplificador de microondas
																\item[o] Hardware criogénico: solo un amplificador y algunos
																				cables coaxiales}
												\end{itemize}
												\begin{center}
																\includegraphics[width=\textwidth]{fdm1}
												\end{center}
								\end{column}
								\begin{column}{0.5\textwidth}
												\begin{center}
																\includegraphics[width=\textwidth]{fdm2}
												\end{center}
								\end{column}
				\end{columns}

\end{frame}

%------------------------------------------------------------------------------
%\section{Firmware: Channelizer}

\begin{frame}{Desarrollo del sistema de excitación/lectura}
\framesubtitle{Canalizador DFT (Channelizer)}
				\begin{columns}
								\begin{column}{0.4\textwidth}
												\begin{itemize}
																\item[*] Implementación m\'as eficiente en t\'erminos de uso de hardware en FPGA
																\item[*] Ancho de banda similar para todos los canales 
																%\item[*] Extensible a un n\'umero grande de canales ($>=1000$)
																\item[*] Posibilidad de realizar el diseño en multi-etapas
												\end{itemize}
								\end{column}
								\begin{column}{0.6\textwidth}

				\begin{center}
								\includegraphics[width=\textwidth]{FDM_channel_diagram}
												\includegraphics[width=\textwidth]{pfb_basic1}
				\end{center}
								\end{column}
								\end{columns}
\end{frame}
%\begin{frame}{Canalizador}
%				\framesubtitle{Operación básica}
%				\centering
%								\includegraphics[width=0.7\textwidth]{pfb_basic1}
%\end{frame}
%\begin{frame}{Canalizador}
%				\framesubtitle{Operación básica}
%				\begin{columns}
%								\begin{column}{0.5\textwidth}
%												\begin{itemize}
%																\item Basado en el concepto de filtro polifásico
%																				\begin{equation*}\label{eq_transformada_z}
%																								X(z) = \sum_{n = -\infty}^{\infty}x(n)\,z^{-n}
%																				\end{equation*}
%
%
%												\end{itemize}
%								\end{column}
%								\begin{column}{0.5\textwidth}
%												\centering
%												\includegraphics[width=0.7\textwidth]{pfb_basic1}
%								\end{column}
%				\end{columns}
%																				\begin{equation*}\label{eq_transformada_z_polifasica}
%																								X(z) = \sum_{\lambda =
%																								0}^{M-1}z^{-\lambda}\,X_\lambda^{(p)}(z^{M})
%																				\end{equation*}
%																				{\color{blue}\begin{equation*}\label{eq_x_polifasica}
%																								X_\lambda^{(p)}(z^{M}) = \sum_{m =-\infty}^{\infty}x(mM+\lambda)\,z^{-mM}
%																				\end{equation*}}
%
%
%\end{frame}
%\begin{frame}{Canalizador}
%				\framesubtitle{Operación básica}
%				\begin{center}
%								\only<1>{\includegraphics[width=0.8\textwidth]{PFB_basic_operation}}
%				\end{center}
%								\begin{center}
%												\only<2>{\includegraphics[width=0.5\textwidth]{PFB_opening_after_modulation}}
%								\end{center}
%\end{frame}
%\begin{frame}{Channelizer operación básica}
%				\begin{center}
%								\only<1>{\includegraphics[width=0.8\textwidth]{PFB_distribution_filter_branches}}
%				\end{center}
%								\begin{center}
%												\only<2>{\includegraphics[width=0.8\textwidth]{PFB_opening_after_modulation}}
%								\end{center}
%\end{frame}
%
%\begin{frame}{Channelizer operación básica}
%				\begin{center}
%								\only<1>{\includegraphics[width=0.8\textwidth]{PFB_distribution_filter_decimators}}
%				\end{center}
%								\begin{center}
%												\only<2>{\includegraphics[width=0.8\textwidth]{PFB_distribution_filter_interpolators}}
%								\end{center}
%\end{frame}
%
%\begin{frame}{Channelizer operación básica}
%				\begin{center}
%								\only<1>{\includegraphics[width=0.6\textwidth]{PFB_distribution_modulators}}
%				\end{center}
%								\begin{center}
%												\only<2>{\includegraphics[width=0.8\textwidth]{PFB_final_arquitecture}}
%								\end{center}
%\end{frame}

%\begin{frame}{Channelizer Tx}
%				%\begin{columns}
%				%				\begin{column}{0.65\textwidth}
%				\begin{center}
%								\only<1>{\includegraphics[width=0.8\textwidth]{Tx_channelizer}}
%				\end{center}
%				%				\end{column}
%				%				\begin{column}{0.65\textwidth}
%								\begin{center}
%												\only<2>{\includegraphics[width=0.8\textwidth]{Tx_channelizer_con_IFFT}}
%								\end{center}
%								%				\end{column}
%								%\end{columns}
%\end{frame}
%\begin{frame}{Channelizer Rx}
%				%\begin{columns}
%				%				\begin{column}{0.65\textwidth}
%				\begin{center}
%								\only<1>{\includegraphics[width=0.8\textwidth]{Rx_channelizer}}
%				\end{center}
%				%				\end{column}
%				%				\begin{column}{0.65\textwidth}
%								\begin{center}
%												\only<2>{\includegraphics[width=0.8\textwidth]{Rx_channelizer_con_IFFT}}
%								\end{center}
%								%				\end{column}
%								%\end{columns}
%\end{frame}
%------------------------------------------------------------------------------
%\section{Comparación de arquitecturas de canalización de gran ancho de banda}

%\begin{frame}{Digital Down-Converter (DDC)}
%				\begin{center}
%								\only<1>{\includegraphics[width=0.9\textwidth]{c1_single_channel_DDC}}
%								\only<2>{\includegraphics[width=0.9\textwidth]{readout3_compa}}
%				\end{center}
%				%\begin{center}
%				%				\only<2>{\includegraphics[width=0.5\textwidth]{nikel_readout2}}
%				%				\only<2>{\includegraphics[width=0.5\textwidth]{nikel_readout3}}
%				%\end{center}
%								Desventaja: los DDC son ``intensivos en silicio'', difícil de
%								incluir muchos ($>1000$) canales en FPGA
%\end{frame}
%\begin{frame}{Arquitectura Pipelined Frequency Transform (PFT)}
%				\begin{center}
%								\only<1>{\includegraphics[width=0.45\textwidth]{readout4_compa}}
%								\only<1>{\includegraphics[width=0.35\textwidth]{readout5_compa}}
%				\end{center}
%
%				División de bandas de frecuencia $\to$ cada etapa sucesiva de la PFT
%				aumenta el número de bandas en un factor de dos
%
%				Desventaja: para un gran número de canales ($>1000$), el árbol se vuelve
%				muy grande. Por ejemplo, 1024 canales requerirían 2046 módulos (CDC o
%				CUC) complejos
%\end{frame}
%
%\begin{frame}{Arquitectura DFT Polifase}
%				\begin{center}
%								\only<1>{\includegraphics[width=0.47\textwidth]{readout6_compa}}
%								\only<1>{\includegraphics[width=0.50\textwidth]{readout7_compa}}
%				\end{center}
%
%				$M = K \to$ críticamente muestreado
%
%				$M < K (I = K/M) \to$ caso sobre-muestreado (oversampled)
%
%				Para nuestra aplicación $I=2$ es una elección \alert{adecuada y
%				suficiente}
%
%				Arquitectura conveniente para cuando el número de canales es grande
%
%\end{frame}
%
%\begin{frame}{Canalizador FFT}
%				\centering
%								\only<1>{\includegraphics[width=0.5\textwidth]{readout2_tabla_compa}}
%				\only<2>{\begin{overpic}[width=0.5\textwidth]{readout2_tabla_compa}
%								\put(0,16){\color{red}\rule{180pt}{1pt}}
%								\put(0,32){\color{red}\rule{180pt}{1pt}}
%								\put(0,16){\color{red}\rule{1pt}{30pt}}
%								\put(99,16){\color{red}\rule{1pt}{30pt}}
%				\end{overpic}}
%
%				\tiny{\textbf{Lillington, John. ``Comparison of wideband channelisation
%				architectures'' (2003)}}
%
%				\only<2>{\normalsize{Para grandes cantidades de bines, el DFT Polifase gana
%				rápidamente, particularmente en términos de memoria y se convierte en
%				la opción preferida para bancos de filtros fijos}}
%\end{frame}

%------------------------------------------------------------------------------
%\section{Opciones de readout}

%\begin{frame}{NIKEL}
%				\begin{center}
%								\only<1>{\includegraphics[width=0.5\textwidth]{nikel_readout}}
%				\end{center}
%								\begin{center}
%												\only<2>{\includegraphics[width=0.5\textwidth]{nikel_readout2}}
%												\only<2>{\includegraphics[width=0.5\textwidth]{nikel_readout3}}
%								\end{center}
%\end{frame}
%
%\begin{frame}{Channelizer operación básica}
%				\begin{center}
%								\only<1>{\includegraphics[width=0.6\textwidth]{PFB_distribution_modulators}}
%				\end{center}
%								\begin{center}
%												\only<2>{\includegraphics[width=0.8\textwidth]{PFB_final_arquitecture}}
%								\end{center}
%\end{frame}

%\section{Vivado}
%\begin{frame}{Vivado}
%
%				Actualmente trabajando con la versi\'on 2016.4
%
%				Cores propios, ej. RxChannelizer y TxChannelizer, etc.
%
%				Mezcla de scripts TCL y Linux embebido
%				\begin{center}
%								\includegraphics[width=0.8\textwidth]{pfb_repo}
%				\end{center}
%\end{frame}
%
%\begin{frame}{Proyectos Vivado}
%
%				%\begin{columns}
%				%\begin{column}{0.45\textwidth}
%				\begin{center}
%								\only<1>{\includegraphics[width=0.95\textwidth]{rxchan16_test2_vivado_project}}
%				\end{center}
%				%\end{column}
%				%\begin{column}{0.45\textwidth}
%								\begin{center}
%												\only<2>{\includegraphics[width=0.8\textwidth]{channelizer16_core}}
%								\end{center}
%								%\end{column}
%								%\end{columns}
%\end{frame}

%\section{Amplificador de bajo ruido}
%\begin{frame}{Amplificador}
%				\begin{columns}
%								\begin{column}{0.45\textwidth}
%												\includegraphics[angle=-90,width=0.62\textwidth]{amp_low_temp1} \\ 
%												\includegraphics[angle=-90,width=0.6\textwidth]{sim900_mainframe_med_temp}
%								\end{column}
%								\begin{column}{0.45\textwidth}
%												\includegraphics[angle=-90,width=0.62\textwidth]{acople_linea_mkid} \\ 
%												\includegraphics[angle=-90,width=0.6\textwidth]{criostato_modelo}
%								\end{column}
%				\end{columns}
%\end{frame}
%\begin{frame}{Amplificador}
%				\begin{columns}
%								\begin{column}{0.45\textwidth}
%												\includegraphics[angle=-90,width=0.62\textwidth]{mkid1} \\ 
%												\includegraphics[angle=-90,width=0.6\textwidth]{mkid2}
%								\end{column}
%								\begin{column}{0.45\textwidth}
%												\includegraphics[angle=-90,width=0.62\textwidth]{mkid3_magnetic_case} \\ 
%												\includegraphics[angle=-90,width=0.6\textwidth]{mkid4}
%								\end{column}
%				\end{columns}
%\end{frame}
%
%\section{Amplificador de bajo ruido}
%\begin{frame}{Amplificador}
%				\begin{columns}
%								\begin{column}{0.45\textwidth}
%												\includegraphics[angle=-90,width=0.62\textwidth]{amp_low_temp1} \\ 
%												\includegraphics[angle=-90,width=0.6\textwidth]{sim900_mainframe_med_temp}
%								\end{column}
%								\begin{column}{0.45\textwidth}
%												\includegraphics[angle=-90,width=0.62\textwidth]{acople_linea_mkid} \\ 
%												\includegraphics[angle=-90,width=0.6\textwidth]{criostato_modelo}
%								\end{column}
%				\end{columns}
%\end{frame}
%\section{Equipamiento BT}
%Using section and subsection commands, outside of frames, provides a table of contents and progress information to beamer.
%\begin{frame}{Equipos BT}
%				\begin{columns}
%								\begin{column}{0.45\textwidth}
%												\includegraphics[angle=-90,width=0.62\textwidth]{IMG_20190523_105037648} \\ 
%												\includegraphics[angle=-90,width=0.6\textwidth]{IMG_20190523_105054061}
%								\end{column}
%								\begin{column}{0.45\textwidth}
%												\includegraphics[angle=-90, width=0.62\textwidth]{IMG_20190523_105443289} \\ 
%												\includegraphics[angle=-90,width=0.6\textwidth]{IMG_20190523_105117465}
%								\end{column}
%				\end{columns}
%
%
%				%				Aim for five to ten slides for a 25~minute presentation.
%				%
%				%				Certainly no more than 15.
%				%
%				%				Use a note form for the content of each slide.
%				%
%				%				\begin{theorem}
%				%								Mathematics works within the Beamer class, $\exp(i\pi)+1=0$\,, including theorems.
%				%				\end{theorem}
%				%
%				%				Click: \url{http://www.maths.adelaide.edu.au} 
%\end{frame}

%\begin{frame}{Equipos BT}
%				\begin{columns}
%								\begin{column}{0.45\textwidth}
%												\includegraphics[angle=-90,width=0.62\textwidth]{IMG_20190523_111114131} \\ 
%												\includegraphics[angle=-90,width=0.6\textwidth]{IMG_20190523_105054061}
%								\end{column}
%								\begin{column}{0.45\textwidth}
%												\includegraphics[angle=-90,width=0.42\textwidth]{IMG_20190523_105647049} \\ 
%												\hspace{2mm}\includegraphics[angle=-90,width=0.42\textwidth]{IMG_20190523_112752228} \\
%												\includegraphics[angle=-90,width=0.42\textwidth]{IMG_20190523_104038976}
%								\end{column}
%				\end{columns}
%				%				Aim for five to ten slides for a 25~minute presentation.
%				%
%				%				Certainly no more than 15.
%				%
%				%				Use a note form for the content of each slide.
%				%
%				%				\begin{theorem}
%				%								Mathematics works within the Beamer class, $\exp(i\pi)+1=0$\,, including theorems.
%				%				\end{theorem}
%				%
%				%				Click: \url{http://www.maths.adelaide.edu.au} 
%\end{frame}

%				\begin{frame}{Amplificador criog\'enico de bajo ruido}
%								\framesubtitle{LNF-LNC03\_14A}
%								\begin{columns}
%												\begin{column}{0.40\textwidth}
%																\hspace{10mm}\includegraphics[width=0.95\textwidth]{lnf-lnc03_14sa}
%												\end{column}
%												\begin{column}{0.60\textwidth}
%																\begin{itemize}
%																				\item Ancho de Banda RF: 0.3-14 GHz
%																				\item Ruido: 4.1 K (típico)
%																				\item Ganancia: 42 dB
%																				\item Potencia DC: Vc= 0.7 V @ 14 mA
%																				\item Conectores RF: G3PO Macho
%																				\item Conector DC: Nano Strip 5 pines hembra
%																\end{itemize}
%												\end{column}
%								\end{columns}
%				\end{frame}

%------------------------------------------------------------------------------
%\section{Hardware}
%\subsection{ZCU111 Ev. Kit}%es el diseño de Israel
%\begin{frame}{Kit de evaluación ZCU111}
%				\centering
%												\includegraphics[width=0.4\textwidth]{isra_mkid1}
%												\includegraphics[width=0.32\textwidth]{isra_mkid2}
%\end{frame}
%\begin{frame}{Barrido en potencia}
%				\centering
%												\includegraphics[width=0.4\textwidth]{isra_mkid3}
%												\includegraphics[width=0.32\textwidth]{isra_mkid4}
%\end{frame}
%\begin{frame}{Barrido en potencia}
%				\centering
%												\includegraphics[width=0.4\textwidth]{isra_mkid1}
%												\includegraphics[width=0.32\textwidth]{isra_mkid2}
%\end{frame}
%

%------------------------------------------------------------------------------
%\section{Diseño del filtro prototipo}
\begin{frame}{Arquitectura}
								\includegraphics[width=0.6\textwidth]{PFB_final_arquitecture}
				\begin{columns}
								\begin{column}{0.4\textwidth}
												\begin{itemize}
																\item[o] Concepto de filtros polifásicos
																\item[o] Extensible a un n\'umero grande de canales ($>=1000$)
																\item[o] Uso de herramientas comunes en DSP (FIR, FFT, etc.)
												\end{itemize}
								\end{column}
								\begin{column}{0.6\textwidth}
								\includegraphics[width=\textwidth]{chann16_out_1_8}
								\end{column}
								\end{columns}
\end{frame}

%\begin{frame}{Arquitectura}
%				\begin{center}
%								\includegraphics[width=0.8\textwidth]{PFB_final_arquitecture}
%				\end{center}
%\end{frame}

\begin{frame}{Filtro PFB - Ch5 $f_\text{centro}= 512\,\text{MHz}$}
				\begin{columns}
								\begin{column}{0.5\textwidth}
												\alert{Primera medición con el algoritmo funcionando en
												RFSoC}
								\end{column}
								\begin{column}{0.5\textwidth}
				\includegraphics[width=0.95\textwidth]{ch5_design}
								\end{column}
				\end{columns}
				%\vspace{-1.2cm}
				%\begin{center}
				%				\only<2>{\includegraphics[width=0.8\textwidth]{ch5_out_456}}
				%\end{center}
				%\vspace{-1.8cm}
				\begin{columns}
								\begin{column}{0.5\textwidth}
												\begin{center}
																\only<1>{\includegraphics[width=0.95\textwidth]{ch5_out_456}}
												\end{center}
								\end{column}
								\begin{column}{0.5\textwidth}
												\begin{center}
																\only<1>{\includegraphics[width=0.95\textwidth]{ch5_design_zoom}}
												\end{center}
								\end{column}
				\end{columns}
				%\vspace{-.8cm}
				%\begin{center}
				%				\only<4>{\includegraphics[width=\textwidth]{ch5_zoom_456}}
				%\end{center}
\end{frame}
\begin{frame}{Ganancia del DAC en RFSoc}
				\framesubtitle{Trabajo en proceso}
				\begin{columns}
								\begin{column}{0.5\textwidth}
												\begin{center}
								\includegraphics[width=\textwidth]{ch5_zoom_456_2}
												\end{center}
								\end{column}
								\begin{column}{0.5\textwidth}
												\begin{center}
								\includegraphics[width=1.2\textwidth]{dac_gain}
												\end{center}
								\end{column}
				\end{columns}
\end{frame}

%\section{El algoritmo de Goertzel}
%\begin{frame}{El algoritmo de Goertzel}
%				\begin{itemize}
%								\item Finalize performance requirements for firmware 
%								\item Finalize performance requirements for Tx and Rx parts 
%								\item First tests of RxChannelizer (front-end, mixers, filters,
%												DC-block, etc.)
%				\end{itemize}
%								\centering
%								\includegraphics[width=0.45\textwidth]{goertzel_algo}
%\end{frame}
%
%\begin{frame}{El algoritmo de Goertzel}
%				\begin{itemize}
%								\item Finalize performance requirements for firmware 
%								\item Finalize performance requirements for Tx and Rx parts 
%								\item First tests of RxChannelizer (front-end, mixers, filters,
%												DC-block, etc.)
%				\end{itemize}
%								\centering
%								\includegraphics[width=0.45\textwidth]{goertzel_non_integer_figure1}
%\end{frame}
%
\section{Conclusiones}
\begin{frame}{Conclusiones}
				\begin{itemize}
								\item Armado de la configuración experimental para la medición
												de MKIDs de uso científico
								\item Establecimiento de colaboraciones con grupos de FNAL y BT
								\item Se especificaron los factores determinantes para la
												excitación y lectura de los detectores MKIDs
								%\item Caracterización de MKIDs de uso científico
								\item Primeras pruebas del algoritmo de canalización en FPGA
								\item L. H. Arnaldi, ``Implementation of an AXI-compliant
												lock-in amplifier on the red pitaya open source
												instrument,'' in IEEE Eight Argentine Symposium and
												Conference on Embedded Systems (CASE) (IEEE, 2017), pp.
												1-6
								\item L. H. Arnaldi, ``Implementation of a Polyphase Filter Bank
												Channelizer on a Zynq FPGA,'' in Press. Argentine
												Conference on Electronics (CAE2020)
				\end{itemize}
\end{frame}
\section{Trabajo futuro}
\begin{frame}{Trabajo futuro}
				\begin{itemize}
								%\item Dise\~no de resonador en Sonnet
								%\item Primeras mediciones a $T_{amb}$ y $T \sim 100\,mK$
								\item Definici\'on de requerimientos para crio (cables,
												conectores, espacios, etc.)
								\item Primeras pruebas con canalizador en Bariloche (pruebas de
												front-end, mezcladores, filtros, DC-block, etc.)
								\item Pruebas con el algoritmo de Goertzel para detección de
												tonos
				\end{itemize}
\end{frame}
%\section{Prefer titles that make a statement}
%\begin{frame}{Prefer titles that make a statement}
%
%				Many opt for meaningless titles and section titles.  
%
%				Instead, make titles convey information.
%
%				\pause
%
%				Use \texttt{pause} commands almost anywhere to progressively step through material.
%
%				\pause
%
%				\begin{figure}
%								\centering
%								\caption{Figure and table environments also work.
%								Use \texttt{includegraphics} or \texttt{pgfplots}.}
%								\begin{tikzpicture}
%												\begin{axis}[footnotesize,axis lines=middle
%																,xlabel={$t$},no marks,thick,domain=0:2.2,smooth ]
%																\addplot+[]{1-2*exp(-2*x)+exp(-4*x)};
%																\addlegendentry{$u(t)$};
%																\addplot+[]{1-exp(-4*x)};
%																\addlegendentry{$v(t)$};
%																\addplot+[]{1+2*exp(-2*x)+exp(-4*x)};
%																\addlegendentry{$w(t)$};
%												\end{axis}
%								\end{tikzpicture}
%				\end{figure}
%
%\end{frame}
%
%
%
%
%\section{Conclusion}
%\begin{frame}{Conclusion}
%
%				Finish with your conclusions displayed: \emph{not} a list of references, \emph{nor} a meaningless ``thank you'' slide.
%
%				\vfill
%				\begin{quote}
%								Three rules of public speaking: Be forthright.  Be brief.  Be
%								seated. \hfill(S. Dressel \& J. Chew, 1987)
%				\end{quote}
%\end{frame}




\end{document}

