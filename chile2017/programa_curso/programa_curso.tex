% Exam Template for UMTYMP and Math Department courses
%
% Using Philip Hirschhorn's exam.cls: http://www-math.mit.edu/~psh/#ExamCls
%
% run pdflatex on a finished exam at least three times to do the grading table on front page.
%
%%%%%%%%%%%%%%%%%%%%%%%%%%%%%%%%%%%%%%%%%%%%%%%%%%%%%%%%%%%%%%%%%%%%%%%%%%%%%%%%%%%%%%%%%%%%%%

% These lines can probably stay unchanged, although you can remove the last
% two packages if you're not making pictures with tikz.
\documentclass[11pt]{exam}
\usepackage[spanish]{babel}
\usepackage[utf8]{inputenc}
\RequirePackage{amssymb, amsfonts, amsmath, latexsym, verbatim, xspace, setspace}
\RequirePackage{tikz, pgflibraryplotmarks}

% By default LaTeX uses large margins.  This doesn't work well on exams; problems
% end up in the "middle" of the page, reducing the amount of space for students
% to work on them.

\usepackage[margin=1in]{geometry}


% Here's where you edit the Class, Exam, Date, etc.
%\newcommand{\class}{Sistemas de Telecomunicaciones I}
\newcommand{\class}{Electrónica Rápida para Detectores de Partículas}
\newcommand{\term}{}
\newcommand{\examnum}{}
\newcommand{\examdate}{}
\newcommand{\timelimit}{}

% For an exam, single spacing is most appropriate
\singlespacing
% \onehalfspacing
% \doublespacing

% For an exam, we generally want to turn off paragraph indentation
\parindent 0ex

\begin{document} 

% These commands set up the running header on the top of the exam pages
\pagestyle{head}
\firstpageheader{}{}{}
\runningheader{\class}{\examnum\ - Página \thepage\ de \numpages}{\examdate}
\runningheadrule

\begin{flushleft}
\begin{tabular}{p{2.8in} r l}
\textbf{\class} %& \textbf{Nombre y Apellido:} & \makebox[2in]{\hrulefill}\\
%\textbf{\term} &&\\
%\textbf{\examnum} %&&\\
%\textbf{\examdate} &&\\
%\textbf{Time Limit: \timelimit} & Teaching Assistant & \makebox[2in]{\hrulefill}
\end{tabular}\\
\end{flushleft}
\rule[1ex]{\textwidth}{.1pt}


%This exam contains \numpages\ pages (including this cover page) and
%\numquestions\ problems.  Check to see if any pages are missing.  Enter
%all requested information on the top of this page, and put your initials
%on the top of every page, in case the pages become separated.\\
%
%You may \textit{not} use your books, notes, or any calculator on this exam.\\
%
%You are required to show your work on each problem on this exam.  The following rules apply:\\
%
%\begin{minipage}[t]{3.7in}
%\vspace{0pt}
%\begin{itemize}
%
%\item \textbf{If you use a ``fundamental theorem'' you must indicate this} and explain
%why the theorem may be applied.
%
%\item \textbf{Organize your work}, in a reasonably neat and coherent way, in
%the space provided. Work scattered all over the page without a clear ordering will 
%receive very little credit.  
%
%\item \textbf{Mysterious or unsupported answers will not receive full
%credit}.  A correct answer, unsupported by calculations, explanation,
%or algebraic work will receive no credit; an incorrect answer supported
%by substantially correct calculations and explanations might still receive
%partial credit.
%
%
%\item If you need more space, use the back of the pages; clearly indicate when you have done this.
%\end{itemize}
%
%Do not write in the table to the right.
%\end{minipage}
%\hfill
%\begin{minipage}[t]{2.3in}
%\vspace{0pt}
%%\cellwidth{3em}
%\gradetablestretch{2}
%\vqword{Problem}
%\addpoints % required here by exam.cls, even though questions haven't started yet.	
%\gradetable[v]%[pages]  % Use [pages] to have grading table by page instead of question
%
%\end{minipage}
%\newpage % End of cover page

%%%%%%%%%%%%%%%%%%%%%%%%%%%%%%%%%%%%%%%%%%%%%%%%%%%%%%%%%%%%%%%%%%%%%%%%%%%%%%%%%%%%%
%
% See http://www-math.mit.edu/~psh/#ExamCls for full documentation, but the questions
% below give an idea of how to write questions [with parts] and have the points
% tracked automatically on the cover page.
%
%
%%%%%%%%%%%%%%%%%%%%%%%%%%%%%%%%%%%%%%%%%%%%%%%%%%%%%%%%%%%%%%%%%%%%%%%%%%%%%%%%%%%%%

\begin{questions}

% Basic question
%\addpoints
%\question[10] Differentiate $f(x)=x^2$ with respect to $x$.
\question \textbf{El tubo fotomultiplicador}

Circuito equivalente -- Señal de salida -- La influencia de los capacitores de acople 
-- Consideraciones generales -- Ganancia -- Linealidad -- Modos de operación -- 
Cálculo de la corriente media de ánodo -- Circuitos de polarización: red pasiva y 
red activa.

\question \textbf{Electrónica rápida para el acondicionamiento de señal} 

Introducción -- Adaptación -- Pre-amplificadores -- Amplificadores TIA (VFA y CFA) -- Respuesta en 
frecuencia -- Ganancia -- Reducción de ruido

\question \textbf{Conversión de señal}

Tipos de ADC: ADC flash, ADC de aproximaciones sucesivas, ADC de conversión voltaje-frecuencia, 
ADC de doble envolvente, ADC Sigma-Delta -- Precisión y resolución -- Precisión del 
ADC vs. precisión del sistema -- Histogramas de ruido ADC -- Mejoras en la 
resolución a través del sobremuestreo y el promediado

\question \textbf{Diseño digital en FPGA}

Principios básicos de VHDL -- Diseños jerárquicos HDL --
Linux en Zynq7000 -- Herramientas para el desarrollo con Zynq7000



\question \textbf{Las electrónicas de LAGO}

Las electrónicas actuales de LAGO -- La placa digitalizadora -- El firmware actual --
La electrónica futura de LAGO -- RedPitaya como sistema adquisidor para LAGO -- Repaso a 
través de las herramientas de desarrollo de RedPitaya.

% Question with parts
%\newpage
%\addpoints
%\question Consider the function $f(x)=x^2$.
%\begin{parts}
%\part[5] Find $f'(x)$ using the limit definition of derivative.
%\vfill
%\part[5] Find the line tangent to the graph of $y=f(x)$ at the point where $x=2$.
%\vfill
%\end{parts}
%
%% If you want the total number of points for a question displayed at the top,
%% as well as the number of points for each part, then you must turn off the point-counter
%% or they will be double counted.
%\newpage
%\addpoints
%\question[10] Consider the function $f(x)=x^3$.
%\noaddpoints % If you remove this line, the grading table will show 20 points for this problem.
%\begin{parts}
%\part[5] Find $f'(x)$ using the limit definition of derivative.
%\vspace{4.5in}
%\part[5] Find the line tangent to the graph of $y=f(x)$ at the point where $x=2$.
%\end{parts}



\end{questions}
\end{document}
