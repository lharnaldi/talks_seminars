\ifcase0  % choose 0=slides, 1=article, 2=refart
	 \documentclass[ignorenonframetext,12pt]{beamer}
	 \geometry{paper=a6paper,landscape}
\or\documentclass[a4paper,11pt]{article}
	 \usepackage{url,beamerarticle}
\or\documentclass[a4paper,11pt]{refart}
	 \let\example\relax
	 \usepackage{url,beamerarticle}
\fi

\ifcase0  % choose a theme like these
	 \usetheme{Goettingen}% I recommend
\or\usetheme{Singapore}
\or\usetheme{Boadilla}
\or\usetheme{Pittsburgh}
\or\usetheme{Madrid}
\or\usetheme{Warsaw} % common choice, but often poor
\fi

\usepackage{graphicx,pgfplots,parskip}
\graphicspath{{media/}}



\title{QUBIC mettings} 
\author{Horacio Arnaldi\\
Laboratorio Detecci\'on de Part\'iculas y Radiaci\'on\\
CAB-IB}
\date{10 de Junio 2019}


\begin{document}

\begin{frame}
				\maketitle
\end{frame}


\begin{abstract}
				This abstract, being outside the frame environment, does not appear in the presentation.  Your outline will be the basis for a couple of sentences of talk for each of the following questions:
				\begin{itemize}
								\item What was done?
								\item Why do it?
								\item What were the results?
								\item What do the results mean in theory and/or practise?
								\item What is the reader's benefit?
								\item How can the readers use this information for themselves? 
				\end{itemize}
\end{abstract}


\begin{frame}{Outline}
				\tableofcontents
\end{frame}

\section{Channelizer}
\begin{frame}{Channelizer}

				Implementaci\'on m\'as eficiente en t\'erminos de uso de hardware en FPGA

				Ancho de banda similar para todos los canales 

				Extensible a un n\'umero grande de canales ($>=1000$)

				Posibilidad de realizar el dise\~no en multi-etapas

				\begin{center}
								\includegraphics[width=0.8\textwidth]{FDM_channel_diagram}
				\end{center}
\end{frame}
\begin{frame}{Channelizer Tx}
				%\begin{columns}
				%				\begin{column}{0.65\textwidth}
				\begin{center}
												\only<1>{\includegraphics[width=0.8\textwidth]{Tx_channelizer}}
				\end{center}
				%				\end{column}
				%				\begin{column}{0.65\textwidth}
				\begin{center}
												\only<2>{\includegraphics[width=0.8\textwidth]{Tx_channelizer_con_IFFT}}
				\end{center}
				%				\end{column}
				%\end{columns}
\end{frame}

\begin{frame}{Channelizer Rx}
				%\begin{columns}
				%				\begin{column}{0.65\textwidth}
				\begin{center}
												\only<1>{\includegraphics[width=0.8\textwidth]{Rx_channelizer}}
				\end{center}
				%				\end{column}
				%				\begin{column}{0.65\textwidth}
				\begin{center}
												\only<2>{\includegraphics[width=0.8\textwidth]{Rx_channelizer_con_IFFT}}
				\end{center}
				%				\end{column}
				%\end{columns}
\end{frame}

\section{Vivado}
\begin{frame}{Vivado}

				Actualmente trabajando con la versi\'on 2016.4

				Cores propios, ej. RxChannelizer y TxChannelizer, etc.

				Mezcla de scripts TCL y Linux embebido
				\begin{center}
								\includegraphics[width=0.8\textwidth]{pfb_repo}
				\end{center}
\end{frame}

\begin{frame}{Proyectos Vivado}

				%\begin{columns}
								%\begin{column}{0.45\textwidth}
				\begin{center}
				\only<1>{\includegraphics[width=0.95\textwidth]{rxchan16_test2_vivado_project}}
				\end{center}
								%\end{column}
								%\begin{column}{0.45\textwidth}
				\begin{center}
												\only<2>{\includegraphics[width=0.8\textwidth]{channelizer16_core}}
				\end{center}
								%\end{column}
				%\end{columns}
\end{frame}

\section{Equipamiento BT}
Using section and subsection commands, outside of frames, provides a table of contents and progress information to beamer.
\begin{frame}{Equipos BT}
				\begin{columns}
								\begin{column}{0.45\textwidth}
												\includegraphics[angle=-90,width=0.62\textwidth]{IMG_20190523_105037648} \\ 
												\includegraphics[angle=-90,width=0.6\textwidth]{IMG_20190523_105054061}
								\end{column}
								\begin{column}{0.45\textwidth}
												\includegraphics[angle=-90, width=0.62\textwidth]{IMG_20190523_105443289} \\ 
												\includegraphics[angle=-90,width=0.6\textwidth]{IMG_20190523_105117465}
								\end{column}
				\end{columns}


%				Aim for five to ten slides for a 25~minute presentation.
%
%				Certainly no more than 15.
%
%				Use a note form for the content of each slide.
%
%				\begin{theorem}
%								Mathematics works within the Beamer class, $\exp(i\pi)+1=0$\,, including theorems.
%				\end{theorem}
%
%				Click: \url{http://www.maths.adelaide.edu.au} 
\end{frame}

\begin{frame}{Equipos BT}
				\begin{columns}
								\begin{column}{0.45\textwidth}
												\includegraphics[angle=-90,width=0.62\textwidth]{IMG_20190523_111114131} \\ 
												\includegraphics[angle=-90,width=0.6\textwidth]{IMG_20190523_105054061}
								\end{column}
								\begin{column}{0.45\textwidth}
												\includegraphics[angle=-90,width=0.42\textwidth]{IMG_20190523_105647049} \\ 
												\hspace{2mm}\includegraphics[angle=-90,width=0.42\textwidth]{IMG_20190523_112752228} \\
												\includegraphics[angle=-90,width=0.42\textwidth]{IMG_20190523_104038976}
								\end{column}
				\end{columns}
%				Aim for five to ten slides for a 25~minute presentation.
%
%				Certainly no more than 15.
%
%				Use a note form for the content of each slide.
%
%				\begin{theorem}
%								Mathematics works within the Beamer class, $\exp(i\pi)+1=0$\,, including theorems.
%				\end{theorem}
%
%				Click: \url{http://www.maths.adelaide.edu.au} 
\end{frame}

				\begin{frame}{Amplificador criog\'enico de bajo ruido}
								\framesubtitle{LNF-LNC03\_14A}
								\begin{columns}
												\begin{column}{0.40\textwidth}
				\hspace{10mm}\includegraphics[width=0.95\textwidth]{lnf-lnc03_14sa}
												\end{column}
												\begin{column}{0.60\textwidth}
																\begin{itemize}
																				\item Ancho de Banda RF: 0.3-14 GHz
																				\item Ruido: 4.1 K (típico)
																				\item Ganancia: 42 dB
																				\item Potencia DC: Vc= 0.7 V @ 14 mA
																				\item Conectores RF: G3PO Macho
																				\item Conector DC: Nano Strip 5 pines hembra
																\end{itemize}
												\end{column}
								\end{columns}
				\end{frame}



\section{Perspectivas de trabajo}
\begin{frame}{Perspectivas de trabajo}
				\begin{itemize}
								\item Dise\~no de resonador en Sonnet
								\item Primeras mediciones a $T_{amb}$ y $T \sim 100\,mK$
								\item Definici\'on de requerimientos para crio (cables,
												conectores, espacios, etc.)
								\item Primeras pruebas con RxChannelizer con ITeDA (pruebas de
												front-end, mezcladores, filtros, DC-block, etc.)
				\end{itemize}
\end{frame}
%\section{Prefer titles that make a statement}
%\begin{frame}{Prefer titles that make a statement}
%
%				Many opt for meaningless titles and section titles.  
%
%				Instead, make titles convey information.
%
%				\pause
%
%				Use \texttt{pause} commands almost anywhere to progressively step through material.
%
%				\pause
%
%				\begin{figure}
%								\centering
%								\caption{Figure and table environments also work.
%								Use \texttt{includegraphics} or \texttt{pgfplots}.}
%								\begin{tikzpicture}
%												\begin{axis}[footnotesize,axis lines=middle
%																,xlabel={$t$},no marks,thick,domain=0:2.2,smooth ]
%																\addplot+[]{1-2*exp(-2*x)+exp(-4*x)};
%																\addlegendentry{$u(t)$};
%																\addplot+[]{1-exp(-4*x)};
%																\addlegendentry{$v(t)$};
%																\addplot+[]{1+2*exp(-2*x)+exp(-4*x)};
%																\addlegendentry{$w(t)$};
%												\end{axis}
%								\end{tikzpicture}
%				\end{figure}
%
%\end{frame}
%
%
%
%
%\section{Conclusion}
%\begin{frame}{Conclusion}
%
%				Finish with your conclusions displayed: \emph{not} a list of references, \emph{nor} a meaningless ``thank you'' slide.
%
%				\vfill
%				\begin{quote}
%								Three rules of public speaking: Be forthright.  Be brief.  Be
%								seated. \hfill(S. Dressel \& J. Chew, 1987)
%				\end{quote}
%\end{frame}




\end{document}

